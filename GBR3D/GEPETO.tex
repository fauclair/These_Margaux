\hypersetup{pdfborder=0 0 0}


%-------------------------------------------------------------------------------------
%\section{Comparison of solitary waves and associated signal from in situ data}
\section[A first evaluation of LES with in-situ \& remote observations]{A first evaluation of LES with in-situ \& remote observations}
\label{sectionCampagne}

The pertinence and the accuracy of the high-resolution Large Eddy Simulations performed so far crucially need to be evaluated based on in-situ or remote observations of both the regional and fine scales of the real ocean. The observation of the latter is somehow difficult at least when these fine scales are localized in small, transient spots. In turn, LES can then appear as a well-adapted tool to help designing the campaign.

In the present section, only a selection of in-situ and remote observations of Gibraltar 2020 campaign is studied. While the exploitation of campaign data is incomplete, some observations are still presented to represent the complete work that was carried out during my Ph-D to provide an as-rigorous-as-possible work including both development of LES, investigation of LES dynamics and evaluation with dedicated observations. Further treatment of in-situ observations and the preparation of the Gibraltar 2022campaign are still being carried out.

\subsection{Field campaign Gibraltar 2020 (an overview)}
The field campaign of in situ measurements Gibraltar 2020 has been carried out by SHOM during the fall of 2020 in the Strait of Gibraltar and inthe western part of the Alboran sea aboard the research and survey ship \textit{L'Atalante}. This campaign and the following Gibraltar 2022 campaign are part of the PrometeVs program and LEFE-GEPETO project. On-site measures were taken by ship-based instruments from 8/10/2020 to 20/10/2020. Among those, sampling of the water column at both end of the strait were realized; at the eastern end of the strait on the 14th and 15th October, and at its western end on the 16th of October.

Additionally, five moorings were deployed as presented in table \ref{tab_moor}, locations are also indicated in figure (\noparref{fig_moor}.A2). Mooring M1 is positioned west of the slope of Camarinal Sill. Mooring M3 is placed in the southern deep half of CS whereas M2 is positioned in a shallow area at the center. M4 and M5 are positioned near each other at some distance east of CS. Three of the moorings (M1, M3 and M5) are equipped with CTD sensors to provide hydraulogical characteristics of the water masses, and the other two (M2 and M4) with ADCP sensors to sample the currents in the water column. Sampling frequencies range from a few tens of seconds to one minute.

\begin{table}[!h]
        \centering
        \begin{tabular}{|c|c|c|c|}
                \hline
                Mooring & type & position & date (UTC)\\ 
                 \hline
                M1 & CTD & 35° 55.264'N ; 5° 46.739'W & 8/10/2020 15h - 9/11/2020 12h\\
                M2 & ADCP & 35° 55.761'N ; 5° 45.288'W & 8/10/2020 5h - 17/10/2020 15h\\
                M3 & CTD & 35° 54.719'N ; 5° 44.459'W & 8/10/2020 13h - 22/10/2020 21h\\
                M4 & ADCP & 35° 55.870'N ; 5° 41.020'W & 8/10/2020 7h - 17/10/2020 14h\\
                M5 & CTP & 35° 56.229'N ; 5° 41.026'W & 8/10/2020 9h - 1/11/2020 14h\\
                \hline
        \end{tabular}
        \captionof{table}{Name, type of sensors, coordinates and date of deployment for moorings during Gibraltar 2020 field campaign.}
        \label{tab_moor}
        %\end{minipage}
\end{table}
In section \ref{section_obs_moor}, mooring data from M2, M4 and M5 are analyzed for a first observation period covering the ten-day period 8/10 to 18/10 during which, as indicated in table \ref{tab_moor}, both types of moorings data are available.


\subsection{Insights from LES simulations in preparation of Gibraltar 2020}

The numerical simulations presented in section \ref{sectionSim3D} are based on a high-resolution, non-hydrostatic model. Atmospheric fluxes are neglected as a first step toward realistic, high-resolution Large Eddy Simulation of the region of Gibraltar strait. This simulations provide information on the flow and processes occuring in the strait that were used to eâre the Gibraltar 2020 campaign.

 
\begin{figure}[!h]
% \centering
 \includegraphics[width=\textwidth]{./GBR3D/Fig_Moor.png}
 \caption {(A1) Water column sampling sites fot (B1) and (B2). (A2) locations of moorings deployed during Gibraltar 2020 (black squares), over the map of standard deviations of parameter Q (colorbar) and the location of the hydraulic jumps of w-type and s-type from high-resolution numerical modelling of the strait of Gubraltar, as presented in section \ref{sectionSim3D}. (B1 and B2) $\Theta$-S diagrams for the series of water column sampling carried ou respectively at the western and eastern exit of the Strait, water mass definitions according to \color{red}\citet{najanro_2014}\color{black}.}
 \label{fig_moor}
\end{figure}

The field of standard deviation of parameter Q and the localization of the hydraulic jumps in figure (\noparref{fig_moor}.b) are for instance issued from those simulations. In combination with external restrictions such as the dense maritime traffic, strong currents and steep slopes of the area, such diagnosis and others were studied to chose the mooring deployment as well as the transect plans for the campaign (not shown). As an example, M1 was positioned down the western slope of the sill, i.e. downflow of a potential primary instability generation area (see section \ref{PartDiag3D} and \ref{section3DResFlow} for a discussion of this diagnosis in high-resolution numerical simulation).

Figure (\noparref{fig_SARIES}) features a comparison between a SAR image (figure (A)) of the strait of Gibraltar with the surface signature of a propagating ISW just east of CS, and the corresponding field of norm of the gradient of surface currents in SimIT at the same date (figure (B)), showing a traveling wave in the same vicinity. Whereas the shape of the train itself differs in the model and observed fields, the simulation gives an accurate idea of the propagation speed of ISWs in the strait of Gibraltar. This was used to predict the position of ISW in relation to the tidal cycle as predicted by the spanish institute Puertos del Estado\footnote{http://www.puertos.es/}. The anticipation of position of ISWs train was accurate at least in the strait of Gibraltar itself. In the Alboran Sea, where the influence of the gyre on the form of the wave packet is important, prediction was not as accurate, with time of arrival being greatly delayed compared to our predictions. 

Beyond the propagation speed, the high resolution of the model means that the shape of individual solitary waves is accurate as it propagates. This is used in the following section \ref{section_obs_moor} to help in the interpretation of mooring data from M4 and M5.


\begin{figure}[!h]
% \centering
 \includegraphics[width=\textwidth]{./GBR3D/Comp_SAR_IES.png}
 \caption {(a) Sentinel-1 Synthetic Aperture Radar (SAR) image (12/09/2017 - 6h18pm UTC). (b) Norm of the gradient of surface horizontal velocity (s-1) in the simulation SimIT (12/09/2017 - 6h30pm or t = 35h30 in simulation time) presented in section \ref{sectionSim3D}.}
 \label{fig_SARIES}
\end{figure}


\subsection{Overview of the mesoscale circulation during the observation period}

The in-situ time period covers one (for ship-based instruments and ADCP moorings) or two (for CTD moorings) neap-spring tide cycles. Figure (\noparref{fig_moor_US3}.B) shows the depth-averaged zonal component of the current measured at CS (data from M2 mooring). The measures begin during the neap-tide part of the fortnightly cycle. The west Alboran Gyre was also present in the West Alboran Sea during the field campaign (not shown). 

Figure (\noparref{fig_moor}.B1 and B2) present the $\theta-S$ diagram from ship-based water column sampling. For both figures, each color refers to a different sampling station indicated in figure (\noparref{fig_moor}.A1).

On the west end of the strait (figure (B1)), no Mediterranean water was sampled at the southernmost station and a well-mixed signal could be identified at the northernmost station, delimiting the path of the Mediterranean outflow between 35.7$^{\text{o}}$ and 36$^{\text{o}}$ N. Among the signals of Mediterranean outflow waters, the two northernmost stations that reach depths shallower than 400 m (orange and yellow) show an enhanced mixing with NACW.

On the east end of the Strait (figure (B2)), WMDW can be found at depth for all stations except the northernmost (yellow). For the next two stations south of the latter, as well as the two southernmost stations, WMDW is mixed with intermediate waters. 

The five northernmost stations' surface waters are fresher waters than the SAW signal at the rest of the stations. This could be due to the northern stations being affected by the upwelling from the Iberian coast. The intermediate Mediterranean waters sampled at these stations are also warmer and saltier compared to the signal of the remaining four, which is interpreted as LIW.


\subsection{Solitary waves at M4 and M5 mooring and currents over CS at M2 mooring}
\label{section_obs_moor}

\begin{figure}[!h]
% \centering
 \includegraphics[width=\textwidth]{./GBR3D/US_moorings1.png}
 \caption {timeseries of mooring data over the water column from M2 (upper row), M4 (center row) and M5 (lower row) mooring. The zonal component of currents is represented for M2 and M4 mooring, and the measured salinity at M5 mooring.}
 \label{fig_moor_US1}
\end{figure}

\begin{figure}[!h]
% \centering
 \includegraphics[width=\textwidth]{./GBR3D/US_moorings2.png}
 \caption {same as figure \ref{fig_moor_US1} for different time-periods.}
 \label{fig_moor_US2}
\end{figure}

\begin{figure}[!h]
% \centering
 \includegraphics[width=\textwidth]{./GBR3D/US_moorings3.png}
 \caption {(A1 to A3) Same as figure \ref{fig_moor_US1} but for a different time-period. (B) Time-series of depth-averaged signal of the zonal component of currents from M2 data (in blue the instant signal recorded, in red the 2 minutes average). For each outflow is indicated the type of signal that is observed at M4 and M5 mooring (see text).}
 \label{fig_moor_US3}
\end{figure}

Figures \ref{fig_moor_US1} to (\noparref{fig_moor_US3}.a) present depth-time records of the zonal velocity (for M2 and M4 mooring) and salinity (for M5 mooring) for five different M2 tidal periods. Tilting by the strong currents provoked the depths of the CTD sensors of the M5 mooring to change overtime, sometime loosing the signal from tens to a hundred of meters at the top of the water column (for exemple see figure (\noparref{fig_moor_US2}.A3 at 15hUTC)). Additionnally, note that whereas the whole water column is presented in those figures for the M2 data, only the upper 300 m (of a 500-m-deep water column) are represented here for M4 and M5 data for a better visualization.

Similarly, figure \ref{Fig_moor_USs} presents the zonal velocity and salinity of the upper 300 m of simulated data at a grid point of coordinate (35.937°N;5.706°W), near M4 and M5, from the simulations SimNT (figures \noparref{Fig_moor_USs}A and B), SimIT (C) and SimST (D) of section \ref{sectionSim3D}. Although those simulations cover a different time-period, the simulated fields present similar patterns of internal waves traveling in the water column as the observed data.


\subsubsection{Currents at M2 and M4 mooring}

In the observations of currents made at mooring M2, periods of inflows and outflows can be distinguished respectively as having mostly eastward or westward components over the water column. During inflow periods, there are always at least two hours during which the whole flow measured by the captors is eastward (for example between 10 and 12 hours in figure (\noparref{fig_moor_US1}.A1)). During outflows, the current can be westward at all captors, as is the case in figure (\noparref{fig_moor_US2}.B1) and (\noparref{fig_moor_US3}.A1), but this is not necessarily the case. 

In figures (\noparref{fig_moor_US1}.A1) and (B1), for example, the baroclinic exchange structure of currents is still distinctive during outflows, with a weak eastward flow in the upper 120 m of the water column over a strong westward current. Figure (\noparref{fig_moor_US2}.A1) presents another case for which the flow in the upper water column becomes momentarily weakly westward between t = 7 h and t = 9 h, with a still clear shear interface at 100-m deep.

In the numerical simulations performed in section \ref{sectionSim3D}, an entirely westward flowing water column at M2 mooring corresponds to an area an hydraulic jump is present. In figure (\noparref{fig_moor}.A2), this location corresponds to the upflow area of the two types of hydraulic jumps identified in section \ref{sectionSim3D} (s-jump and w-jump), and depicted respectively as grey and black points.

At mooring M4, the flow of the water column can become unidirectional during both outflow and inflow periods during the spring tide part of the fortnightly cycle. In this occasions, a shear area still subsists that matches with the salinity interface between Mediterranean and Atlantic waters identified at mooring M5 (see for example at t = 14 h in figure (\noparref{fig_moor_US2}.A2) and (A3) at depth ranging between 150 and 200 m).

\subsubsection{Propagation of high frequency waves}

\begin{figure}[!h]
% \centering
 \includegraphics[width=\textwidth]{./GBR3D/US_M4SimMIV.png}
 \caption {time series of salinity (black lines) and zonal velocity (color) in the upper 300 m in simulations SimNT( A and B), SimIT (C) and SimST (D) of section \ref{sectionSim3D} at the gridpoint of coordinates (35.937°N;5.706°W). Abscises is simulation time. }
 \label{Fig_moor_USs}
\end{figure}

It is on the salinity interface observed at M5 mooring that the signal of propagating internal gravity waves can be spotted, sometimes matching with anomalies in the current field of mooring M4.

Figures (\noparref{fig_moor_US1}.B3) at t = 3 h, (\noparref{fig_moor_US2}.A3) at t = 8h30, and (\noparref{fig_moor_US2}.B3) at t = 19h30, show as a recurring feature an abrupt lifting of the interface, that does not appear in the simulations data of figure \ref{Fig_moor_USs}.

Another recurring signal in M5-mooring data are the large amplitude troughs that can be seen during the inflow period of the tidal cycle. In this data set, it appears clearly in observation data made at t = 29 h in figure (\noparref{fig_moor_US2}.B3). In simulation data (for example t = 49 h in figure (\noparref{Fig_moor_USs}.C)), this signal corresponds to a westward traveling train of ISWs that is generated by reflection off the Moroccan coast of the well-known eastward traveling ISWs train that is generated at CS.

The focus is now made on the signal showing up at M4 and M5 mooring, usually 3 hours or sooner after the maximal outflow at M2 mooring. Five distinctive types of signals are identified and categorized with letters o, L, B, S, and 2S :

\begin{itemize}
\item \underline{Linear-internal tide (o), figures (\noparref{fig_moor_US1}.A2-A3)}: the depth of the salinity interface at M5 mooring and maximum shear at M4 mooring evolves linearly, except for some low amplitude traveling waves at the interface in M5 mooring at t = 10h30. At M2 mooring (figure \noparref{fig_moor_US1}.a1), there is a distinctive shear in the water column during the preceding outflow, with slightly positive velocity in the upper layer. This signal is also seen in SimNT as shown in figure (\noparref{Fig_moor_USs}.A).
%
\item \underline{Small-amplitude internal wave (L), figures (\noparref{fig_moor_US1}.B2-B3)}: in the salinity data, there is a signal that looks like two internal waves of relatively small amplitude (10 m) at M5 mooring at t = 4h30. At M4 mooring, the depth of maximum shear of zonal velocity still evolves in a linear manner as in the previous (o) case. At M2 mooring (figure \noparref{fig_moor_US1}.b1), the interface of westward flow evolves at the same depth as in the (o) case but in the upper layer velocity becomes almost nil from t = 1 h to t = 3h30. This signal is also seen in SimNT in figure (\noparref{Fig_moor_USs}.B) at t = 28.5 h of simulation, and is associated there in the velocity field with a mode-1 anomaly.
%
\item \underline{Internal-traveling bore (B), figures (\noparref{fig_moor_US2}.A2-A3)}: at M5 mooring, the salinity interface drops by 50 m at t = 11 h which resembles the signal of a westward-propagating internal bore. At M4 mooring, however, the depth of maximum shear still evolves linearly, but before the arrival of the internal bore signal, the flow in the water column is negative at all depths. At M2 mooring over CS, the upper layer velocity is nil or lightly negative during the outflow. This type of signal is not recovered in the simulations that have been performed.
%
\item \underline{Train of internal-solitary waves (S), figures (\noparref{fig_moor_US2}.B2-B3)}: a succession of 7 troughs passes at M5 mooring starting at t = 24h15. The first one has an amplitude of 80 m. At M4 mooring, this series corresponds to mode-1 anomalies of the velocity field. At M2 mooring, the flow is westward throughout the water column during the preceding outflow, with an abrupt return to a sheared two-layer state at t = 22h30, corresponding to the loss of hydraulic control and the release of the western hydraulic jump over CS. In simulations, this type of signal is seen for instance in simIT and presented in figure (\noparref{Fig_moor_USs}.C) with two troughs at  t = 44 h. In these simulations, this type of signal at mooring M4 and M5 follows the release of a s-jump type of hydraulic jump (i.e., at maximum outflow, the western hydraulic jump is located over the shallowest part of CS).
%
\item \underline{Two close trains of internal-solitary waves (2S), figures (\noparref{fig_moor_US3}.A1-A2)}: five troughs can be seen propagating at M5 mooring starting at t = 2 h, but are not propagatingin order of decreasing amplitude. The first trough has an amplitude of 80 m and is followed by two short-wavelength, small-amplitude troughs. Then at t = 2h30, an over-100-m amplitude trough propagates at M5 mooring. It is in turn followed by a smaller-amplitude trough. The mode-1 anomaly of the velocity field is seen clearly at M4 mooring for the first two waves, then the fourth larger amplitude one. At M2 mooring, as in the previous (S) case, the flow through the water column transitions from wholly westward to sheared two-layer at t = 1 h. In numerical simulation SimST (figure \noparref{Fig_moor_USs}.D), four waves can be identified. They follow this pattern, the first two waves have decreasing amplitude, the third has a larger amplitude than the first two, and the fourth has a smaller amplitude. In this case, this pattern corresponds to two different trains of ISWs. The first (second) train corresponds to the previously  released hydraulic jump east (west) of CS. In the numerical simulations, this signal follows the release of a w-jump (i.e., at maximum outflow the west hydraulic jump is located over the western slope of CS).
\end{itemize}

Both S and 2S signals are linked to westward flow of the whole water column at CS, which should indicate that, as in the numerical simulations, a hydraulic jump was present west of M2 mooring.

The 2S case can be observed in numerical simulations and in figure (\noparref{Fig_moor_USs}.D). The amplitude of the first wave which corresponds to the eastern hydraulic jump of CS can however be very small. While the wave(s) produced by the release of the eastern hydraulic jump are always present, at the latitude of moorings M4 and M5, its amplitude depends (i) on the northern extent of the eastern hydraulic jump at maximum outflow (i.e., how high a latitude it reaches) and (ii) on the initial angle taken by the released non-linear wave as it first propagates in a slightly southern direction.

As the two sets of ISWs propagate further in the strait, the second train overtakes the first one. Indeed the propagation speed of ISWs depends on their amplitude (the larger the faster), so eventually they appear as a merged and sorted train of ISWs. For instance, the "S" structure in simulation appears because the wave released by the western hydraulic jump of CS overtook the eastern one(s) sooner due to their initial closeness.

So although it appears here that two cases are distinct (the S case following an s-jump and the 2S case following a w-jump), there might be a possibility that slowly propagating waves from an s-jump could also appear as a 2S structure at M4 and M5 mooring, and conclusion cannot be reached on the structure of the two hydraulic jumps at CS only on the basis of the signal at M4 and M5 moorings.


\subsection{Transition between outflow types \& ISWs generation in Gibraltar strait}

The classification of the previous section is applied to signals at M4 and M5 mooring following each outflow of the first observation period and is marked as annotations in figure (\noparref{fig_moor_US3}.B).

A pattern emerges linking outflow type and strength of the averaged currents at CS. The beginning of the period corresponds to the neap-tide part of the fortnightly cycle, and either (L) or (o) type of outflows are detected, with no hydraulic jump at CS. The first solitary wave is observed at M4 and M5 mooring the 12/10/2020. Due to the diurnal variation of the M2 tide, the tidal flow over the following period is weaker (less than 1 m/s) and the signal at M4 and M5 moorings is an (o) case.

Except for one specific (B) case (14/10/2020), during the remainder of the period, trains of ISWs (with either a S or 2S structure) are propagating through M4 and M5 mooring. The stronger outflows lead to (2S) signals in agreement with the numerical simulations presented in section \ref{sectionSim3D}. Under especially strong outflows, the internal hydraulic jump generated over CS is swept downstream as a w-jump, resulting in an initially increased distance between the eastern and western jumps. This distance may not be overcome as quickly upon release of the hydraulic jump as in the s-jump case. This explains the distinction between S and 2S cases, however as explained previously, for some outflows, the distinction between the two may remain subjective.

Only one (B) case is observed, it was not featured in numerical simulations so it is less evident whether it can be attributed to the presence of an hydraulic jump over CS. Whereas the variation with depth of currents at M2-mooring site in the preceding outflow shows a shallower interface and more westward currents in the upper layer than for the (o) and (L) cases, it may be more akin to a near supercritical flow regime engendering some form of propagating steepening interfacial disturbance.

It was seen in section \ref{section_sim3D_ISW} that, in numerical simulations, even if no hydraulic jump occurs at CS, the flow of the barotropic tide in the strait of Gibraltar can lead to the steepening of a long interfacial wave that later develops into a train of ISWs. This train contains a lesser number of ISWs than in the release of hydraulic jump case as it propagates toward the Alboran Sea. 

Figure (\noparref{fig_SAROBS}) is a SAR image taken during the Gibraltar 2020 campaign in the morning of October, 9. A curved surface signature of higher reflectivity can be seen in the Alboran Sea, looking like the front of an ISW (for exemple in figure (\noparref{fig_SARIES}.A)). But looking at (\noparref{fig_moor_US3}.B), all preceding outflows are of the "L" case for the signal at M4 and M5 mooring at this date, with no hydraulic jump at CS. The small amplitude internal gravity wave that was observed at M5 mooring, if propagating east, could be responsible for the signal in the Alboran Sea that looks like oe ISW, and is similar to what was encountered in simulations.

\begin{figure}[!h]
% \centering
 \includegraphics[width=0.6\textwidth]{./GBR3D/SAR_OBS_GEPETO.png}
 \caption {Sentinel-1 Synthetic Aperture Radar (SAR) image from 09/10/2020 - 6h18am UTC.}
 \label{fig_SAROBS}
\end{figure}


\subsection{Discussion \& perspectives}

A first confrontation between LES and observations has been carried out showing at least a qualitative agreement. Simililarities are found between simulated fields and four of the five types of signals encountered in data of moorings M4 and M5. Two of those signals, S and 2S are of clear trains of ISWs propagating after the release of the hydraulic jumps at Camarinal Sill that induces a westward flow in all the water column in M2 mooring data. 

For the L signal, the presence of the hydraulic jump is doubtful, but figure (\noparref{fig_SAROBS}) provides one satellite image of what appears to be as lone ISW signal propagating in the Alboran Sea after such an outflow. According to this observation, and the reproduction of this behaviour in numerical simulations, it is possible that the mechanism of release of the hydraulic jump is not the only one responsible for generation of the observed ISWs in the Strait of Gibraltar and in the western part of the Alboran Sea. ISWs trains are indeed observed in other areas of the global ocean, without being linked to the establishment andrelease of an hydraulic jump (see for exemple \citet{chen_2017}).

Looking at the barotropic currents measured at M2, there is a pattern linking the amplitude of the tide to the signal seen at the M4 and M5 moorings. Especially there seems to be a treshold over which hydraulic jumps and solitary waves begin to be observed. The reproduction of this treshold in numerical simulations of the Strait of Gibraltar is expected to be difficult. Even in high-resolution regional modelling such as in section \ref{sectionSim3D}, numerical parameters will influence the transition from a no hydraulic jump regimen to a generation of hydraulic jump one. The state of the stratification, for exemple, will play a part as the condition for Atlantic and Mediterranean layers to be supercritical, and hence on the moment hydraulic jumps begin to appear. This will depend on the quality of the water masses defined in the numerical simulation as well as atmospheric and large-scale forcings. Other important physical factors are the high-sensivity to the bathyetric data and the tidal forcing.

Several improvements have already been made over the simulations of section \ref{sectionSim3D}, but are still beeing evaluated and, as a consequence, have not been included in the present section. 
\begin{itemize}
\item Atmospheric fluxes are specified at the surface of the ocean.  This provides a better representation of the stratification in the upper surface and has important consequences on the characteristics of the pycnocline and thus on the characteristics of the internal waves, bores and solitons.
\item The high-resolution dynamics in the Strait of Gibraltar can now be explicitly simulated by downscalling the regional circulation (from the Gulf of Cadix to midway of the Alboran Sea). A three-step embedding has already been carried out using AGRIF library from 900-m to 60-m resolution simulations through a 180-m resolution implementation.
\end{itemize}
 


\section{Conclusion}

\color{green} Rappel méthode, bilan développement LES, bilan maquette...\color{black}




\section{Annexe : Singular Value Decomposition (SVD)}
\label{annexeSVD}
Singular Value Decomposition (SVD) consists in finding a basis of singular values and orthonormal singular vectors for a G x T matrix A (complex or real), so that :
\begin{equation}
A = U \Sigma V^* 
\end{equation}
Where the G x k matrix $U$ and k x T matrix $V$ (and its conjugate $V^*$) are the left and right singular vectors respectively, and $\Sigma$ is the diagonal k x k matrix of singular values associated to each couple of singular vectors.

To analyze a time-varryinig signal of variable $\psi$ on a 3D grid (or 2D as in section ..), the spatial field of each of the T timestep is appended into one column of length G to create the GxT matrix A, where T are the number of iterations of the 3D field. 

After proceeding with the SVD of this matrix A, the column number i of U is reformed into a 3D field that gives the spatial structure of the field of $\psi$ associated with the i singular value, that has a time-variation in the i row of the singular right vector V.

In section \ref{sectionSim2D}, SVD was applied to the complex field of $\psi=w+iu$. consecutive vectors of the basis with close singular value and that exhibited similar time-variations were combined to ... .In section \ref{sectionSim3D}, SVD is applied to the field of the real quantity $Q$.
