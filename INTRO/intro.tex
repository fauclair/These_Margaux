\hypersetup{pdfborder=0 0 0}

3/4 pages au moins en français (idem pour conclusion)
\begin{itemize}
\item fines échelles et leur rétroaction/structuration de circu générale
\item choix région Gibraltar (Rencontre deux masses d'eaux, alimentation Méditerranée, outflow med)
\item les fines échelles à Gibraltar (quels phénomènes (solitons, ressaut, etc))
\item Outils de la thèse : num, obs (compagne,sat)
\item état des lieux modélisation num fines échelles océaniques, problematique quantification mélange diapycnale
\item plan
\end{itemize}

Certains des éléments bibliograpjiques sont repris dans les introductions des différents chapitres, conçus comme des articles.





Morceaux de intro GBR3D : 

The amplitude of the exchange varies over timescales larger than the semi-diurnal tide. The lower frequencies (whether seasonal or inter-annual) are usually linked to atmospheric forcing over the Mediterranean \citep{sanchez-roman_2012}. The tidal eddy-fluxes have their own variability associated to the spring-tide cycle and to the monthly tides, with for example a greater depth and stronger shear during neap tides, but more intense mixing during spring tide \citep{naranjo_2014,vargas_2006}.\color{red}(enlever? sert à rien? que dans intro plus générale???)\color{black}


\subparagraph{Conclusion générale/dans le manuscrit}
Les simulations ... sont première LES maisblablablablabla (trad ce que avait dit...). Besoin outil diagnostique du mélange, chapitre prochain...