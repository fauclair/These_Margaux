

\section{Première incursion en \textit{Terra Incognita} dans le détroit de Gibraltar}
Durant mes trois années de thèse de doctorat, je me suis attachée à mener à bien une exploration des \textit{fines échelles} dans la région du détroit de Gibraltar. Cette incursion en \textit{Terra Incognita} avaient deux types d'objectifs: sur la dynamique des fines échelles tout d'abord (et la perspective de mieux comprendre les processus de fine échelle dans la région du détroit: ressaut, ondes internes solitaires, grandes structures turbulentes...), mais aussi sur la mise en place, la réalisation et l'évaluation de simulations numériques explicites (dites \textit{LES zonale}) de ces grandes structures turbulentes.

Je me suis par conséquent appuyée sur une simulation numérique explicite et originale des \textit{grandes structures turbulentes} à très haute résolution (seulement quelques dizaines de mètres) dans la région du détroit. La simulation numérique ne pouvant en aucun cas répondre à elle seule à l'ensemble des questions qui se posaient, j'ai effectué des allers-retours entre mes outils numériques et l'observation de l'océan réel. Si la majeure partie de mon travail de thèse a été consacrée à l'élaboration et à la description d'un écoulement océanographique local au moyen de l'outil numérique, j'ai en effet eu l'occasion de toucher à d'autres aspects de l'océanographie physique.
En plus de développements plus conceptuels sur l'évaluation du mélange,  j'ai eu la chance de pouvoir interagir étroitement avec mes collègues du SHOM lors de la préparation, la conduite et l'exploitation d'une campagne de mesures dédiée à la problématique des fines échelles dans la région du détroit de Gibraltar.

\section{Des avancées de plusieurs ordres/ d'ordre numérique}
Avec le soutien du Groupement de Recherche CROCO, j'ai mis en place une hiérarchie de maquettes numériques à très haute résolution dans la région du détroit de Gibraltar : maquettes bi-dimentionnelles (chapitre \noparref{chapGBR2D}) puis tri-dimentionnelles (chapitre \noparref{chapGBR3D}). Le choix assumé avec cette démarche de modélisation a été de proposer des maquettes de plus en plus réalistes, mais sans complexifier trop rapidement la dynamique simulée au détriment de la compréhension des processus et mécanismes complexes rencontrés dans cette région océanique très particulière. 

L'abandon de l'hypothèse hydrostatique (ou des paramétrisations non-hydrostatiques) a été un préalable à la modélisation des fines échelles océaniques. Le passage à une dynamique non-hydrostatique permet en effet de représenter explicitement l'accélération verticales de processus de sous-mésoéchelle comme les courants gravitaires ou les cheminées convectives. Il est de même essentiel pour la simulation explicite de processus tels que les ondes solitaires, pour lesquelles la dispersion non-hydrostatique est un mécanisme clé aux côtés de l'advection non-linéaire. Tous ces exemples sont fondamentalement non-hydrostatiques, et la simulation explicite est particulièrement nécessaire dans l'étude de la propagation des ondes de leur lieu de génération à leur dissipation où elles participent activement au mélange diabatique. %Les deux processus peuvent être très éloignés géographiquement, ce qui rend complexe leur représentation implicite dans les modèles de circulation générale.


%Elle l'est aussi de façon tout aussi évidente pour représenter explicitement l'accélération verticales des courants gravitaires ou des cheminées convectives. Tous ces exemples sont fondamentalement non-hydrostatiques, l'éloignement géographique de leur zone de génération avec la zone dans laquelle ils sont dissipés (et dans lesquelles ils participent donc activement au mélange diabatique) rend complexe leur représentation implicite dans les modèles de circulation générale.

\color{blue}
De tels processus de sous-mésoéchelle peuvent conduire à des instabilités primaires de cisaillement : les grandes structures turbulentes ainsi générées ouvrant la voie à la cascade turbulente directe dite de Kolmogorov. Ces grandes structures turbulentes marquent la limite basse de ce qui est appelé dans ce manuscrit "fines échelles". Elles peuvent intervenir de façon très hétérogènes et intermittentes dans la colonne d'eau, en particulier lorsqu'elles sont liés à l'instabilité de certains processus dits de sous-mésoéchelle, et sont associées à des développements non-hydrostatiques.% rendant incontournable leur représentation explicite et par conséquent la remise en cause des approximations autour de l'hypothèse non-hydrostatique. 

%La \textit{Terra Incognita} est ainsi fondamentalement non-hydrostatique. ......
La simulation non-hydrostatique de la \textit{Terra Incognita} a été possible avec la levée de l'hypothèse de Boussinesq et, avec elle, celle de l'hypothèse incompressible. J'ai ainsi pu simuler un océan à toit libre à des coûts en adéquation avec les moyens de calcul à disposition, jusqu'à une résolution de $25m$ sur l'horizontal lors de tests de convergence. Cette résolution de l'ordre des dizaines de mètres a permis la simulation d'instabilités primaires de cisaillement.

%il m'a été possible de ... avec la levée de l'hypothèse de Boussinesq, en allant jusqu'à une résolution horizontale de 25m à l'horizontale et de quelques mètres à la verticale.

Cependant, les échelles spatio-temporelles auxquelles apparaissent ces instabilités au sein de la colonne d'eau sont très dépendantes des mécanismes entraînant leur génération. L'échelle des instabilités issues des écoulement de sous méso-échelle est généralement éloignés de celle aboutissant à la déstabilisation des couches limites de surface et de fond, dont l'extension spatiale est de l'ordre du mètre ou en-dessous. % (quelques dizaines de mètres dans la région de Gibraltar par exemple contre seulement quelques mètres, voire moins, dans les couches limites de surface et de fond).
Ainsi, pour les couches limites de surface et de fond, la modélisation (implicite) de ces "grandes" structures demeure indispensable à ce jour. L'étude réalisée des fines échelles correspond donc à une LES zonale, approche pertinente étant donnés les outils numériques et les moyens de calcul à disposition.\\
%Ceci tend à faire de la simulation LES zonale une approche pertinente avec les outils numériques et les moyens de calcul à disposition.

%La représentation explicite de tels processus de sous-mésoéchelle n'est toutefois pas encore la raison majeure pour laquelle l'évolution des codes numériques d'océan s'impose comme une condition sine qua none pour une entrée en \textit{Terra Incognita}. Nous avons vu en effet que les processus de sous-mésoéchelle conduisaient, ou plus exactement pouvaient conduire à des instabilités primaires de cisaillement, les grandes structures turbulentes ainsi générées ouvrant la voie à la cascade turbulente directe dite de Kolmogorov. Nous avons aussi, au besoin, confirmé que ces instabilités pouvaient intervenir de façon très hétérogènes et tout aussi intermittentes dans de la colonne d'eau. 
%les échelles spatio-temporelles auxquelles apparaissent ces instabilités au sein de la colonne d'eau sont très dépendantes des mécanismes entraînant leur génération et sont généralement éloignées de celles aboutissant à la déstabilisation des couches limites de surface et de fond (quelques dizaines de mètres dans la région de Gibraltar par exemple contre seulement quelques mètres, voire moins, dans les couches limites de surface et de fond).% Une conséquence est qu'au sein de la colonne d'eau, les échelles caractéristiques des grandes structures turbulentes sont du même ordre de grandeur que certains processus dits de sous-mésoéchelle rendant incontournable leur représentation explicite et par conséquent la remise en cause des approximations autour de l'hypothèse non-hydrostatique. La \textit{Terra Incognita} est fondamentalement non-hydrostatique. \\

%La levée de l'hypothèse de Boussinesq et, avec elle, celle de l'hypothèse incompressible nous a permis de simuler un océan à toit libre à des coûts en adéquation avec les moyens de calcul dont nous disposions (nous avons par exemple pu descendre à des résolutions de seulement 25 m dans la région du détroit lors de nos tests de convergence).\\

%La dynamique des fines échelles (associant sous-mésoéchelle et micro-échelle) nous a par conséquent permis d'avancer dans notre réflexion sur la nécessité ou non de basculer en \textit{LES} pour aborder la simulation de la sous-mésoéchelle océanique puisque la représentation explicite des grandes structures turbulentes s'impose de fait dans la colonne d'eau partout où le spectre des processus de sous-mésoéchelles ne peut être distingué de celui de ces grandes structures turbulentes et donc de l'amorce de la cascade turbulente. Il ne s'agit en effet plus de simuler des processus sous-mailles conduisant au mélange mais bien de représenter les processus d'instabilités conduisant aux plus grandes structures turbulentes. Dans les couches limites de surface et de fond, la modélisation (implicite) de ces "grandes" structures (d'échelles potentiellement plus réduites) demeurent cependant indispensables à ce jour. Ceci tend à faire de la simulation LES zonale une approche pertinente avec les outils numériques et les moyens de calcul à disposition aujourd'hui.

Les premières évaluations de la dynamique de fine échelle explicitement simulée dans les maquettes numériques mises en place au cours de ma thèse sont très encourageantes et offrent un premier état des lieux de notre capacité à explorer cette \textit{Terra Incognita} océanique. La confrontation tant aux observations satellites qu'à une première campagne d'observation in-situ et aux marégraphes dans la région du détroit, a montré que les processus "clés" de la dynamique du détroit étaient correctement représentés et localisés (aussi bien spatialement que temporellement): ressauts hydrauliques ou encore génération, propagation mais aussi réflexion des ondes solitaires de grande amplitude. Un mécanisme de génération secondaire pour les trains d'ondes solitaires (s'ajoutant donc à celui lié aux ressauts hydrauliques) a de surcroît été mis en évidence.

\color{blue}L'étude de la structuration complexe du jet méditerranéen en amont et en aval du détroit, ou celle des caractéristiques des grandes structures turbulentes, demeurent encore exploratoires en ce sens qu'une évaluation précise et vraisemblablement stochastique est à ce jour nécessaire (et en partie planifiée dans le cadre du second volet de la campagne in situ Gibraltar 2022).\color{black}\\

%Entre le passage de la 2D a la 3D, voit que relaxation de l'hypothèse hydrostatique et plus raffinement d'échelle permet mieux simuler les solitons. A l'occasion de la campagne gib2020, voit que même si période différente, cette génération toujours en place et trouve résultat proches aspect train d'ondes solitaires. DE plus, pense posibiloité régime secodnaire de générationd e solitons au détroit autre que ressaut hydraulique.
Il a aussi été confirmé que la seule implantation d'une maquette numérique non-hydrosta\-tique ne suffit pas à la bonne représentation de la dynamique locale du détroit. La dynamique des fines échelles est évidemment très dépendante de la bonne représentation de la bathymétrie dans la région du détroit (avec toutes les difficultés inhérentes à l'obtention d'une bathymétrie de quelques dizaines de mètres de résolution) mais aussi de son immersion dans une dynamique régionale (à mésoéchelle) réaliste. Concernant la bathymétrie je tiens à remercier le SHOM pour les données bathymétriques de grande qualité sur lesquelles j'ai pu appuyer mes travaux. Concernant la dynamique régionale de moyenne échelle, je souhaite aussi remercier l'ENEA (et en particulier G. Sannino) pour avoir fourni des sorties hydrostatiques du modèle MIT-GCM pour cette région dans le cadre d'une collaboration entamée avec le post-doctorat de L. Bordois. Je tiens enfin à remercier le groupement MERCATOR pour les données de circulation générale qui ont été utilisées pour divers tests réalisés dans le cadre des travaux de l'équipe.

En ce qui concerne le "forçage" par la circulation de mésoéchelle, un travail de raffinement de la dynamique est encore en cours de déploiement (voir les perspectives ci-après). Un autre point méritant une attention particulière est la simulation de la propagation de l'onde de marée, élément essentiel de la dynamique de fine échelle dans la région du détroit de Gibraltar. En effet, la dynamique de la marée dans cette région de faible étendue jouxtant deux bassins océaniques aussi différents que l'Atlantique Nord et la Méditerranée, relève du véritable challenge. Les premières confrontations aux observations de marégraphes et la bonne représentation des processus que cette marée induit dans le détroit (ressauts hydrauliques ou encore ondes solitaires...) sont très encourageantes. Toutefois un raffinement progressif de la marée semble inéluctable pour espérer aller plus loin en termes de réalisme et de précision des processus induits (voir de nouveau les perspectives ci-après).

%Il faut cependant souligner que l'implémentation d'une grille haute résoluation a gibraltar en ayant a dispositoion code non-hydro ne suffit pas a avoir bonne dynamique. Role tres important avoir une bonne bathymétrie, et en plus d'une bonne initialisation des masses d'eaux,  a vu que à Gibraltar fine échelle suit la marée. En particulier cette dynamique de fine échelle va avoir transition d'un régime à l'autre abrupte lié à la mise en place contrôle hydraulique. Prise en compte de la marée nécessite simulation parent/grande emprise pour avoir propagation onde de marée dans cette région jouxtant deux bassins océaniques.

%Cependant , avoir ce code à disposition ne suffit pas pour être appliqué a une grille fine et avoir une dynamique correcte,. E(avec des transitions abruptes liés au controle hydraulique présent ou non entre régimes de généartion des solitons) et la prise en compte de ce forçage nécessite une simulation parent pour sa bonne propagation dans cette zone enter deux bassins. (ou plus tard cette dernière phrase:)D'autres développements en cours sur le raffinement d'échelle, avec en plus intégration du forçage atmosphérique

%Cette simulation non-hydrostatiques a haute résolution s'accompagnait aussi dans les deux cas (2D et 3D) de la simulation explicite des grandes structures turbulentes qui marquent l'entrée de la cascade directe, est bien en terre inconnue...
%\color{red} Ici, résultats numériques...\\
%De point de vue de la dynamique des fines échelles, nos allers-retours entre simulation numérique et observation nous ont permis de...\\
%La campagne de mesures Gibraltar 2020 a en particulier été l'occasion de ...\\
%\color{black}
%\color{red}(Margaux : Bof je ne sais pas trop comment les placer ici, les verrous sont adressés après avec les perspectives pour moi...)

%Plusieurs \textit{verrous}, tant dynamiques que numériques ont été pour certains confirmés pour d'autres identifiés au cours de mes travaux. 
% résumé rapide des verrous cités en introduction...\\

\section{La question du mélange turbulent en \textit{Terra Incognita}}
Cette première exploration des grandes structures turbulentes à Gibraltar a confirmé la présence d'instabilités de type Kelvin-Helmholtz à la fois très localisées (à l'interface entre les jets méditerranéens et atlantiques) et intermittentes (bien que très corrélées avec le cycle de marée). J'ai pu montrer que leurs caractéristiques étaient très sensibles au forçage par la marée et ce, de manière fondamentalement non-linéaire puisque liée à la mise en place des ressauts hydrauliques. D'un point de vue numérique cette fois, j'ai aussi mis en évidence leur sensibilité à la modélisation des processus sous-mailles.
Dans la mesure où ces structures sont les portes d'entrée de la cascade turbulente directe, leur simulation explicite ainsi qu'une représentation statistiquement correcte de l'amorce de ces cascades menant au mélange est un enjeu "clé" de notre entrée en \textit{Terra Incognita}.


Une question sous-jacente une fois ces grandes structures turbulentes simulées est celle de la quantification du mélange turbulent. Durant ma dernière année de thèse, j'ai mené à bien un travail de fond visant à généraliser l'approche basée sur l'équation d'évolution de l'énergie potentielle disponible (et par là même de l'énergie potentielle de fond) issue des travaux de \cite{lorenz_available_1955}. Ce travail, présenté en chapitre \ref{chapBPE} de ce manuscrit, a montré la nécessité de reformuler ce bilan dans le contexte de colonnes océaniques de volume variable. L'introduction d'un terme lié aux mouvements de la surface libre s'est en particulier révélé indispensable dans la série de configurations tests que j'ai proposée dans ce chapitre. % J'ai aussi proposé une généralisation de l'algorithme proposé par \cite{winters_available_1995} afin de prendre en compte une bathymétrie variable et des champs de densité en coordonnées "s". 
Cette étude demeure évidemment exploratoire, et les choix algorithmiques doivent encore être affinés pour déboucher sur une localisation et une quantification du mélange turbulent dans un contexte pleinement réaliste impliquant en particulier une analyse à très haute résolution sur des régions étendues. Le choix de grandeurs physiques permettant de rendre compte correctement de l'évolution très hétérogène et instationnaire du mélange est encore ouvert à ce jour.
%En particulier le choix d'une quantité homogène et instantanée pour représenter le mélange qui fait intervenir une cascade d'échelles spatiales mais aussi temporelles, ne semble pas adapté.  Le mélange $=$ phénomène progressif à plus d'un titre?

Un autre verrou, peut-être plus conceptuel, est lié à la nécessité de converger sur une "grandeur" (ou un jeu de "grandeurs" liées les unes aux autres) qui serait étudiée dans des contextes aussi différents que l'observation in-situ, la simulation physique (en laboratoire) et la simulation numérique, tout en pouvant être mis en relations. Cette grandeur doit aussi prendre tout son sens et être partagée par les schémas de fermeture turbulente des modèles de circulation générale.
%Encore plus loin, on peut se demander si retour de ces études sur les paramétrisations des modèles globaux, y compris climatiques, qui ne peuvent pas se permetter de tels raffinement d'échelles. Pour phénomène localisé comme l'écoulement autour du seuil de camarinal (aussi localisé avec réponse au forçage de marée plus ou moins prévisible), encore possible, mais quid des déferlements d'ondes internes de gravité?


%qui va provenir de progressifs et localisés dans la colonne d'eau et dans régions particulières 

%Quelques éléments des chapitres précédents :

%Modèle d'océan capable de simuler les fines échelles en représentant leur dynamique de manière correcte. Avec la simulation des grandes structures turbulentes à l'entrée de la cascade directe, choix de la fermeture turbulente. Comment choisir ? Se poser la question de ce que représente le mélange / comment il est représenté, un diagnostique, et surtout qui pourra être confronté à des observations de terrains pour affiner la démarche.
%Simulations physiques peuvent apporter une part de réponse, et besoin d'observations in situ.

%\color{red} Tu as raison... un petit point plus spécifiquement sur le mélange semble être une bonne idée ici... Tu peux expliquer le retour à théorie que tu as opéré (cas volume évoluant au cours du temps), ce que tu as proposé et ce qu'il reste d'après toi à faire pour parvenir une localisation et quantification précise du mélange.\\
%Je suis d'accord sur le fait que simulation physique et campagnes dédiées seront absolument essentielles! Un mot peut-être aussi effectivement sur ce que ton approche peut et doit apporter pour l'amélioration des fermetures turbulentes (avec quoiqu'il arrive des limitations liées à la modélisation (implicite) ces processus. Gibraltar est un bon exemple... tu ne pourras jamais paramétrer par exemple le mélange induit à des dizaines de kilomètres par un soliton que tu ne représentes pas... Idem pour les marées internes générées par exemple sur une dorsale océanique et mélangeant à des centaines de kilomètres de leur lieu de génération lors de leur déferlement (expérience américaine HOME autour de la dorsale d'Hawaï).\\
%\color{black}


\section{Perspectives pour les prochaines incursions en \textit{Terra Incognita}}
%Comme a été rappelé, dynamqiue a Gib besoin info grande échelle, travaux sur raffinement d'échelle qui intègre forçage atmosphérique et qui permettra aussi voir rétroaction sur les grandes échelles.
%A vu dynamique sensible a facteur physiques et numériques (phénomène stochastique).... tests de sensibilité ont un coût de calcul, même si peut être réduit par avancés comme hétérogènes GPU/CPUs, sorties de grandes grilles, est bien dans le Big Data, en parallèle de l'avancée vers des diagnostiques du mélange plus poussé(cf paragraphe précédent), sera peut-être accompagné de méthodologie issue de l'IA.
Une conclusion de mes travaux qui s'impose est la nécessité de penser la simulation des \textit{fines échelles océaniques} dans le cadre d'approches par raffinement local. Ceci est essentiel aussi bien pour le forçage par la marée que pour la circulation régionale à mésoéchelle dans le détroit de Gibraltar. En parallèle de mes travaux de thèse, l'équipe CROCO du LAERO a dors et déjà mis en place une maquette basée sur trois niveaux de raffinement pour aboutir à une résolution de quelques dizaines de mètres dans le détroit à partir d'une circulation régionale kilométrique. Des allers-retours entre les simulations numériques sur lesquelles ces maquettes débouchent et l'océan réel sont actuellement en cours dans le cadre des projets in-situ SHOM Prometevs - LEFE Gepeto Gibraltar 2020-2022.

De plus, au cours de mes travaux de thèse, seul l'effet de la fine échelle sur la composition des masses d'eaux a été abordé, mais la question de l'impact du mélange ou de la friction sur la distribution des traceurs passifs biogéochimiques \citep{penney_2020} et sur la structuration de la vorticité potentielle \citep{morel_potential_2019} demeure encore ouverte.

Cette thèse propose une simulation originale des grandes structures turbulentes dans un contexte réaliste. Je me suis pour cela appuyée sur un outil de modélisation nouveau au développement duquel j'ai activement participé dans le cadre de mes travaux de thèse: le code CROCO à toit libre non-hydrostatique et compressible \citep{hilt_2020}. Outre la simulation par raffinement locale, cinq pistes d'évolutions sont aujourd'hui soit en cours d'étude, soit encore à l'état de projet. Il est tout d'abord indispensable de penser les schémas sous-mailles dans le contexte d'une LES dite zonale \cite{friess_modelisation_2010}. Cela implique en particulier des allers-retours avec des simulations physiques d'une part et avec l'observation in-situ d'autre part. Dans ce contexte, la convergence de la LES zonale vers des simulations directes (dites DNS) est indispensable et doit être démontrée. Les performances de l'outil numérique doivent évidemment être accrues et de l'importance des gains réalisés dépendra la richesse de la dynamique simulée dans les prochaines années. Le code CROCO dans sa version non-hydrostatique et compressible a pour cela été porté sur la future génération de processeurs hétérogènes CPU / GPU. La compressibilité et plus spécifiquement déjà la propagation acoustique des anomalies de pression est actuellement en cours d'évaluation dans l'outil numérique CROCO en temps que processus\footnote{Thèse de doctorat de Pierre-Antoine Dumont, SHOM.}: une telle étude va permettre de mieux comprendre la dynamique compressible sous-jacente et donc de mieux l'appréhender dans le contexte des fines échelles océaniques. Enfin une approche stochastique est actuellement lancée dans le cadre du Groupement de Recherche CROCO \citep{memin_fluid_2014}, la simulation des grandes structures turbulentes devra être pensée ou plus exactement repensée dans un tel contexte.
%ont montré que le mélange était finalement capable de structurer les plus grandes échelles. Ces auteurs ont en effet mis en évidence l'apparition à plus grande échelle de relations linéaires entre la masse volumique et les traceurs passifs qu'ils simulaient.\color{black}
%Aussi effet sur PV, et lien avec intéraction avec le fond, qui dans LES ici encore LES zonale reste paramétrisée.

Enfin, l'exploration des fines échelles initiée durant ma thèse de doctorat est focalisée sur la région du détroit de Gibraltar, qui s'apparente à un écoulement bicouche dans une zone à bathymétrie accidentée. Les outils numériques développés ou en cours de développement ouvrent d'intéressantes perspectives d'applications à d'autres zones d'intérêt, comme les zones de convection profonde, d'autres détroits, des régions d'upwelling côtier ou encore les régions littorales \citep{marchesiello_tridimensional_2021}...



%En particulier, un intérêt de ces études. Le fait que la zone est peu étendue et la dynamique est prévisible en suivant le régime de marée en fait un bon endroit pour des observation, ce ne sera pas le cas de toutes les zones, surtout pour l'océan profond difficile d'accès, et l'évaluation du réalisme de tels configurations peut être plus indirect.

%\color{blue}
%raffinement de la dynamique (imbrication numérique...)\\
%perspectives pour la partie numérique: efficacité, techniques numériques, processeurs hétérogènes (CPU-GPU)...\\
%application à d'autres régions...\\
%\color{black}
%Thèse s'inscrit dans les développements numériques, si de tels avaient déjà été entamé dans les études atmosphériques, les plus petites échelles impliquées dans l'océan et le besoin de code non-hydrostatique capables de représenter cette dynamique couplet au problème de la surface libre ont été un frein. Les puissances de calcul de plus en plus développées ont permis avec cette thèse 

%D'autres offres de calcul plus performant avec les GPUs

%Avec ces nouvelles capacités de calcul, l'intérêt a été d'explorer ce que peut donner ce code, avec l'exploration de l'écoulement dans le détroit de gibraltar. 









