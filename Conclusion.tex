
\color{blue}
\section{Première incursion en \textit{Terra Incognita} dans le détroit de Gibraltar}
Durant mes trois années de thèse de doctorat, je me suis attachée à mener à bien une première \textit{exploration des fines échelles} dans la région du détroit de Gibraltar. Les objectifs de cette première incursion en \textit{Terra Incognita} étaient de deux ordres:
objectifs sur les dynamiques des fines échelles tout d'abord, avec l'idée de mieux comprendre les processus de fine échelle dans la région du détroit (ressaut, ondes internes solitaires, grandes structures turbulentes...) d'une part, mais aussi objectifs numériques, avec la mise en place, la réalisation et l'évaluation de simulations numériques explicites (dite \textit{LES zonale}) de ces grandes structures turbulentes.\\
Je me suis par conséquent appuyée sur une simulation numérique explicite et originale des \textit{grandes structures turbulentes} à très haute résolution (seulement quelques dizaines de mètres) dans la région du détroit. La simulation numérique ne pouvant évidemment en aucun cas répondre à l'ensemble des questions qui se posaient, j'ai effectué des allers-retours entre mes outils numériques et l'observation de l'océan réel. J'ai en effet eu la chance de pouvoir interagir étroitement avec mes collègues du SHOM lors de la préparation, la conduite et l'exploitation d'une campagne de mesures dédiée à la problématique des fines échelles dans la région du détroit de Gibraltar durant ma dernière année de thèse.\\
Plusieurs \textit{verrous}, tant dynamiques que numériques ont été pour certains confirmés pour d'autres identifiés. 
\color{red} résumé rapide des verrous cités en introduction... \color{blue}\\

\section{Des avancées de plusieurs ordres}
Avec le soutien du Groupement de Recherche CROCO, j'ai mis en place une hiérarchie de maquettes numériques à très haute résolution dans la région du détroit de Gibraltar, maquettes bi-dimentionnelles (\S ) puis tri-dimentionnelles (\S ). Le choix assumé avec cette démarche de modélisation était de proposer des maquettes de plus en plus réalistes sans pour autant "brûler les étapes" en complexifiant trop rapidement la dynamique simulée au détriment de la compréhension des processus et autres mécanismes complexes rencontrés dans cette région océanique très particulière. \color{red} Ici, résultats numériques...\\
De point de vue de la dynamique des fines échelles, nos allers-retours entre simulation numérique et observation nous ont permis de...\\
La campagne de mesures Gibraltar 2020 a en particulier été l'occasion de ...\\
\color{black}

\section{Le mélange turbulent en \textit{Terra Incognita}?}
Quelques éléments des chapitres précédents :

Modèle d'océan capable de simuler les fines échelles en représentant leur dynamique de manière correcte. Avec la simulation des grandes structures turbulentes à l'entrée de la cascade directe, choix de la fermeture turbulente. Comment choisir ? Se poser la question de ce que représente le mélange / comment il est représenté, un diagnostique, et surtout qui pourra être confronté à des observations de terrains pour affiner la démarche.
Simulations physiques peuvent apporter une part de réponse, et besoin d'observations in situ.

\color{red} Tu as raison... un petit point plus spécifiquement sur le mélange semble être une bonne idée ici... Tu peux expliquer le retour à théorie que tu as opéré (cas volume évoluant au cours du temps), ce que tu as proposé et ce qu'il reste d'après toi à faire pour parvenir une localisation et quantification précise du mélange.\\
Je suis d'accord sur le fait que simulation physique et campagnes dédiées seront absolument essentielles! Un mot peut-être aussi effectivement sur ce que ton approche peut et doit apporter pour l'amélioration des fermetures turbulentes (avec quoiqu'il arrive des limitations liées à la modélisation (implicite) ces processus. Gibraltar est un bon exemple... tu ne pourras jamais paramétrer par exemple le mélange induit à des dizaines de kilomètres par un soliton que tu ne représentes pas... Idem pour les marées internes générées par exemple sur une dorsale océanique et mélangeant à des centaines de kilomètres de leur lieu de génération lors de leur déferlement (expérience américaine HOME autour de la dorsale d'Hawaï).\\
\color{black}


En particulier le choix d'une quantité homogène et instantanné pour représenter un mélange qui va provenir de progressifs et localisés dans la colonne d'eau et dans régions particulières 

\section{Perspectives}

\color{blue}
raffinement de la dynamique (imbrication numérique...)\\
perspectives pour la partie numérique: efficacité, techniques numériques, processeurs hétérogènes (CPU-GPU)...\\
application à d'autres régions...\\
\color{black}
Thèse s'inscrit dans les développements numériques, si de tels avaient déjà été entamé dans les études atmosphériques, les plus petites échelles impliquées dans l'océan et le besoin de code non-hydrostatique capables de représenter cette dynamique couplet au problème de la surface libre ont été un frein. Les puissances de calcul de plus en plus développées ont permis avec cette thèse 

D'autres offres de calcul plus performant avec les GPUs

Avec ces nouvelles capacités de calcul, l'intérêt a été d'explorer ce que peut donner ce code, avec l'exploration de l'écoulement dans le détroit de gibraltar. 


Ici seul l'effet de la fine échelle sur la composition des masses d'eaux a été abordé, mais turbulence effet sur traceurs géochimiques, et autre effet dynamique non négligeable est la modification du contenu en PV.


D'autres zones d'intérêt, comme les zones de convection profonde, d'autres détroit, des régions d'upwelling côtier, etc. En particulier, un intérêt de ces études. Le fait que la zone est peu étendue et la dynamique est prévisible en suivant le régime de marée en fait un bon endroit pour des observation, ce ne sera pas le cas de toutes les zones, surtout pour l'océan profond difficile d'accès, et l'évaluation du réalisme de tels configurations peut être plus indirect.

