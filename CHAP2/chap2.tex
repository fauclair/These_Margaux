\hypersetup{pdfborder=0 0 0}



\section{Conservation equations for the Ocean}

Conservation of a quantity is noted ...

\subsection{Conservation of mass, salt, heat (ou plusieurs sous-parties?)}

continuity equation

\subsection{Conservation of momentum}

\subsubsection{General expression}

Conservation of momentum in the ocean :


Navier stokes equations, fluide newtonien



\subsubsection{Dissipation, direct turbulent cascade, mixing (Ou section tout court???)}

This is accomplished by molecular friction in Navier-Stokes equations. 

\subsubsection{Discretization and parametrisation in numerical models}

RANS(Reynolds Averaged Navier Stokes)/LES(Large eddy simulation)/DNS(direct numerical simulation)

In the case of RANS and LES, turbulent closure escheme are used to parametrize the subgrid dissipative processes.

GLS (for Generic Length Scale) schemes/methods parametrize turbulent flux/reynolds terms/eddy viscosity and diffusivity by solving two (local) evolution equations, one for the turbulent kinetic energy $k$, another for a second parameter, for exemple the rate of dissipation $\epsilon$ in the case of k-$\epsilon$ closure.

Smagorinsky 

Parle smago et GLS...

Aussi... dissipation associée a la discrétisation...


\subsection{Conservation of Background Potential Energy}

\subsubsection{Definition}
Potential energy in the ocean is 

\subsubsection{}



\section{CROCO}



