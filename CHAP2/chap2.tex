
\section{Résumé du chapitre en français}
Le présent chapitre présente de façon détaillée le \textit{modèle d'océan} utilisé dans le cadre de ma thèse pour simuler numériquement les grandes échelles turbulentes (ou LES$^{\noparref{LES}}$) dans la région du détroit de Gibraltar mais aussi, plus généralement, pour développer des modèles analytiques simplifiés de processus au service de cette approche numérique. Plusieurs sections du chapitre ont été intégrées à des publications acceptées \citep{hilt_2020}, \citep{auclair_modied_2021} ou en cours de rédaction  \citep{auclair_NBQ1_2021} mais aussi au rapport d'études du programme amont PROTEVS Gibraltar du SHOM \citep{auclair_modelisation_2019}. L'ensemble du chapitre est par conséquent rédigé en anglais.

Dans une première partie (\S \ref{section_prim_eq}) sont introduites les équations de conservations usuelles de l'océanographie physique, dont les équations de Navier-Stokes, point de départ du développement du \textit{modèle d'océan}. Le choix est ensuite fait de se placer dans le contexte d'une grille verticale curviligne qui permet d'épouser la forme des fonds marins et de suivre les mouvements de la surface libre de l'océan (dont un certain nombres de développements ont aussi présentés en annexe (\noparref{section_annexe2})). Dans ce cadre, l'expression de l'évolution de l'énergie potentielle gravitationnelle (PE) et de ses sous-compartiments, énergie potentielle disponible (APE) et énergie potentielle de fond (BPE), est développée dans un volume local d'océan (section \ref{section_PE_chap2}).

En particulier, l'évolution de la BPE fait apparaître un terme source lié aux mouvements de la surface libre. Comme l'évolution de la BPE est lié au mélange diapycnale, la bonne expression de son bilan est impérative afin de faire des diagnostics de quantifications de mélange se basant sur cette méthode.

Dans la deuxième partie du chapitre (section \ref{section_croco}), est présenté le fonctionnement du code communautaire à cœur non-hydrostatique, compressible et à surface libre CROCO, basé sur le \textit{modèle d'océan} de la première partie. 
L'implémentation numérique de ce \textit{modèle d'océan} a demandé d'importants développements tant algorithmiques que numériques, développements qui ne peuvent être menés à bien s'ils miment directement la physique de l'océan. Parce qu'il est très général, ce \textit{modèle d'océan} peut réaliser la synthèse de processus dynamiques dans une gamme très étendue d'échelles spatio-temporelles depuis la circulation basse fréquence, jusqu'aux ondes acoustiques. Ce sont plus spécifiquement les plus fines échelles et les plus hautes fréquences qui peuvent imposer les plus fortes restrictions à l'approche numérique envisagée ; ce sont donc les processus associés et en particulier les processus ondulatoires acoustiques ou gravitaires qui ont été étudiés en priorité. 

%Ce \textit{modèle d'océan} est suffisamment général pour autoriser la représentation explicite d'une large gamme de processus allant des ondes et modes acoustiques dans un océan compressible aux structures turbulentes fondamentalement non-hydrostatiques de \textit{fine échelle} (associées par exemple à des instabilités de Kelvin-Helmholtz) en passant par des processus ondulatoires internes de grande amplitude (tels que les solitons).

J'ai participé à une partie des développements de CROCO durant ma thèse, sur le plan purement numérique tout d'abord, avec l'implémentation et l'évaluation de nouveaux schémas numériques dans un contexte pleinement réaliste et la mise en œuvre de stratégies originales pour la LES. Sur le plan de la dynamique océanique ensuite, avec la réalisation d'études de processus de fines échelles et l'étude des interactions complexes entre ces processus.

%Ce \textit{modèle d'océan} a de plus servi de base au développement dans le cadre de ma thèse de diagnostiques originaux dédiés à la simulation des grandes échelles turbulentes océaniques (LES) dans un contexte réaliste: évaluation quantitative du mélange turbulent (\S \noparref{chapter_bpe}), mise en évidence et caractérisation de ces structures, études de ressauts hydrauliques (\S \noparref{PartDiag3D})...

En parallèle des développements numériques et des travaux sur la dynamique de la région du détroit de Gibraltar menés dans le cadre de la présente thèse de doctorat, a été développé et publié un modèle analytique suffisamment général pour décrire la dispersion des ondes et des modes acoustiques, des ondes et des modes internes de gravité ou encore des ondes de gravité de surface \citep{auclair_modied_2021}. Le modèle analytique de dispersion a de plus été utilisé pour explorer la dynamique ondulatoire dans la région du détroit de Gibraltar.
J'ai participé et co-signé cette étude en support du développement numérique de CROCO, étude qui n'a pas été incluse dans le présent manuscrit.

Dans ce qui suit du présent chapitre, c'est l'anglais qui est utilisé pour les raisons évoquées précédemment.

%Dans une première partie, les équations analytiques servant de base à ce \textit{modèle d'océan} sont présentées (\noparref{section_prim_eq}), l'approche numérique choisie et co-développée pour le coeur numérique non-hydrostatique, compressible et à surface libre de CROCO est détaillée en partie \ref{section_croco}. Un certain nombre de développements sont enfin présentés dans les annexes (\noparref{section_annexe2}).


\section{A non-hydrostatic, compressible, free-surface ocean model}
\label{section_prim_eq}

%%%%%%%%%%%%%%%%%%%%%%%%%%%%%%%%%%%%%%%%%%%%%%%%%%%%%%%%%%%%%%%%%%%%%%%%%%%%%
 %----------------------------------------------------------------------------
 \subsection{Continuous free-surface compressible equations in z-coordinates}
 %----------------------------------------------------------------------------
\label{subsectiongenesystem}
\subsubsection{Model equations in conservative form}
Conservation of mass, conservation of momentum (Newton's second law of motion), conservation of total energy (first law of thermodynamics) and conservation of any tracers are the backbones of ocean dynamics. In the ocean, the conservation of mass can be written as a prognostic equation for density (written $\rho$), the conservation of momentum leads to prognostic equations for the three components of momentum (written $\rho \mathbf{v}$) and the conservation of total energy (or first law of thermodynamics) can be stated as a prognostic equation for potential temperature ($\theta$). The conservation of chemical species can then be expressed as a prognostic equation for salinity ($S$). These conservation equations consequently lead to the following general system of prognostic equations (expressed in flux form):
\begin{subequations}
 \begin{alignat}{2}
 \displaystyle
 \label{NS_a} 
 & \frac{\partial\rho}{\partial t} &&= - \mathbf{\nabla}\cdot(\rho \mathbf{v})\\[3mm]  
 \label{NS_b}
 & \frac{\partial \rho \mathbf{v}}{\partial t} 
	 &&= -\mathbf{\nabla}\cdot(\rho \mathbf{v}\otimes \mathbf{v}) 
	  \color{black} -2\rho\ \mathbf{\Omega}\ \times \ \mathbf{v} \color{black} -\mathbf{\nabla}p + 		
	\mathbf{\nabla}\cdot\left(
	\mu(\mathbf{\nabla}\mathbf{v}+\mathbf{\nabla}\mathbf{v}^{\ T})
 +\mu_2(\mathbf{\nabla}\cdot\mathbf{v})\ \mathbf{I}\ \right)
 +\rho \mathbf{g}\\
 %
 \label{NS_c}
 & \frac{\partial \rho \theta}{\partial t} &&=-\mathbf{\nabla}\cdot(\rho \theta\mathbf{v})
 +\mathbf{\nabla}\cdot\color{black}(\kappa_\theta\mathbf{\nabla}{\theta})\color{black}\\[3mm]
 %
 \label{NS_d}
 & \frac{\partial \rho S}{\partial t} &&=-\mathbf{\nabla}\cdot(\rho S\mathbf{v})
 +\mathbf{\nabla}\cdot\color{black}(\kappa_S\mathbf{\nabla}{S})\color{black}
 %
  \end{alignat}
\end{subequations}
with $\mu$, $\mu_2$, $\kappa_T$ and $\kappa_S$ respectively the dynamical and bulk viscosities and the thermal and salt diffusivities. $\mathbf{\Omega}$ is the earth instant rotation vector.
Assuming that variables are in thermodynamic equilibrium, the equation of state (EOS) can be formulated as a non-linear, diagnostic functional relation between temperature, salinity, density and (total) pressure (written $p$):
\begin{equation}
 \label{NS_e}
 \rho = \rho_{eos}[\theta,S,p]
\end{equation}

\subsubsection{Boundary conditions}
The position of the interface separating the ocean and the atmosphere must additionally be calculated and is introduced as a boundary condition. This can be achieved by stating that a salty-water particles that is just bellow this interface in the ocean, remains at the interface, leading to the surface kinematic relation:
\begin{equation}
  \displaystyle
  \label{NS_BC2}
  %\frac{\textrm{d}\zeta(\mathbf{x}_{\scriptscriptstyle H},t)}{\textrm{dt}}=w(\mathbf{x}_{\scriptscriptstyle H},z=\zeta)
  \frac{\partial \zeta}{\partial t}=w(\mathbf{x}_{\scriptscriptstyle H},z=\zeta)-\mathbf{v}_H(\mathbf{x}_{\scriptscriptstyle H},z=\zeta)\cdot\mathbf{\nabla}_H\zeta
\end{equation}
where $\zeta$ is the free-surface anomaly in the vicinity of the geoid and subscribe $H$ indicates that only the horizontal component is considered. Assuming then that ocean water cannot penetrate the ocean bottom (at depth $z=-H$):
\begin{equation}
 \displaystyle
 \label{NS_BC0}
  \mathbf{v}(\mathbf{x}_{\scriptscriptstyle H},z=-H)=\mathbf{0}
\end{equation}
Neglecting surface-tension pressure drop, the boundary condition for pressure at the surface of the ocean is given by:
\begin{equation}
 \displaystyle
 \label{NS_BC1}
  p(\mathbf{x}_{\scriptscriptstyle H},z=\zeta,t)= p_{atm}
\end{equation}
with $p_{atm}$ the atmospheric pressure above the surface of the ocean.
The resulting system of prognostic equations, diagnostic relations and boundary conditions leads to a non-linear problem whose main characteristics is the wild spectrum of dynamic processes involved (see for instance \cite{gill_atmosphere-ocean_1982} or \cite{vallis_atmospheric_2006}). Periodic processes such as ocean waves can give a comprehensive overview of the extension of space-time spectrum of transient processes which can propagate in the ocean and \cite{auclair_modied_2021} derive a compressible, free-surface, stratified model of two dispersion relations for wave-numbers and pulsation gathering acoustic, surface and internal waves and insisting on the modification of the dispersion of gravity (acoustic) waves by compressibility (gravity and stratification).

Formulated thus, the system of Navier-Stokes and conservation equations for a free-surface ocean can, at least in theory, be integrated straightforwardly. All variables but the pressure have their own prognostic equation and pressure can be diagnosed from the EOS \ref{NS_e}. Note that the system can be reformulated so that pressure is also given by a prognostic equation.

\subsubsection{Evolution of the density field}
For a linear approximation of the equation of state, a simple evolution equation of $\rho$ can be obtained as a combination of equations \ref{NS_c} and \ref{NS_d} leading to:
\begin{equation}
\displaystyle
\frac{d \rho}{d t}=
%\frac{\partial}{\partial x} \bigg(\kappa_\rho^h \frac{\partial \rho}{\partial x}\bigg\rvert_{tz}\bigg)_{tz}
%+ \frac{\partial}{\partial z} \bigg( \kappa \frac{\partial \rho}{\partial z}\bigg\rvert_{tx}\bigg)_{tx} 
 \mathbf{\nabla}\cdot\color{black}(\kappa_{\rho} \mathbf{\nabla}{\rho})
\label{eq_diff_cart}
\end{equation}
where $\kappa_{\rho}$ is the equivalent diffusivity of density.

 %----------------------------------------------------------------------------  
 %\subsection{Density and pressure decomposition}
 %----------------------------------------------------------------------------
 
\subsection{Terrain-following coordinates}
\label{subsection_scoord}

 %%%%%%%%%%%%%%%%%%%%%%%%%%%%%%%%%%%%%%%%%%%%%%%%%%%%%%%%%%%%%%%%%%%%%%%%%%%%%
\subsubsection{Definition}
%%%%%%%%%%%%%%%%%%%%%%%%%%%%%%%%%%%%%%%%%%%%%%%%%%%%%%%%%%%%%%%%%%%%%%%%%%%%%
The capacity of numerical models to mimic the evolution of global or regional oceanic circulation relies on horizontal and vertical definition of the grid on which the Navier-Stokes and conservation equations previously defined are solved and integrated in time.

Due to considerations of the representation of bathymetric features and free-surface evolutions, terrain-following coordinates, or S-coordinates, are chosen for the vertical discretisation. They are generally defined on generalized constant-$s$ surfaces with $s$ given by:
\begin{equation}
 \displaystyle
 s=s(x,y,z,t)=s(\mathbf{x},t)
\end{equation}
requiring thus that $s$ be a monotonic function of the vertical coordinate $z$:
\begin{equation}
 \displaystyle
 \frac{\partial s}{\partial z}\bigg\vert_{xyt}\ne 0
\end{equation}
$\partial s / \partial z$ is continuous and single-signed (either strictly positive or negative).

%%%%%%%%%%%%%%%%%%%%%%%%%%%%%%%%%%%%%%%%%%%%%%%%%%%%%%%%%%%%%%%%%%%%%%%%%%%%%
\subsubsection{Examples}
%%%%%%%%%%%%%%%%%%%%%%%%%%%%%%%%%%%%%%%%%%%%%%%%%%%%%%%%%%%%%%%%%%%%%%%%%%%%%
Several examples and comparisons on the choice of $s(\mathbf{x},t)$ are given in chapter 6 of \citet{griffies_fundamentals_2004}.
Following  \citet{shchepetkin_regional_2005},  less general $\sigma$-coordinates can be defined by:
\begin{equation}
 \displaystyle
 z(\mathbf{x},\sigma,t)=\sigma H(\mathbf{x_h})\quad or\quad z(\mathbf{x},\sigma,t)=\sigma(H(\mathbf{x_h})+\zeta(\mathbf{x_h},t))+\zeta(\mathbf{x_h},t)
\end{equation}
%or:
%\begin{equation}
% \displaystyle
% z(\mathbf{x},\sigma,t)=\sigma(H+\zeta)+\zeta
%\end{equation}
where $H(\mathbf{x_h})=H(\mathbf{x,y})$ is the bottom topography and $\zeta(\mathbf{x_h},t)$ the surface elevation anomaly. Its generalization to s-coordinates is defined by:
\begin{equation}
 \displaystyle
 z(\mathbf{x},s,t)=\mathcal{S}(s) H(\mathbf{x_h})
\end{equation}
which is currently written:
\begin{equation}
 \displaystyle
 z(\mathbf{x},\sigma,t)=\mathcal{S}(\sigma) H(\mathbf{x_h})
\end{equation}
and $S(\sigma)$ can be a non-linear function. Some current definitions are presented on the Wiki-Roms web-site \footnote{\url{https://www.myroms.org/wiki/Vertical_S-coordinate}}.


%%%%%%%%%%%%%%%%%%%%%%%%%%%%%%%%%%%%%%%%%%%%%%%%%%%%%%%%%%%%%%%%%%%%%%%%%%%%%
\subsubsection{Vertical velocities}
%%%%%%%%%%%%%%%%%%%%%%%%%%%%%%%%%%%%%%%%%%%%%%%%%%%%%%%%%%%%%%%%%%%%%%%%%%%%%
The definition of such a new vertical coordinate requires the derivation of the associated vertical velocity at the grid point. Using the coordinate transformation presented in section \ref{annexe_coordS} of appendix \ref{annexe_ocmod},  $w \equiv v_z$ can be decomposed as :
\begin{subequations}
  \begin{alignat}{2}
  \displaystyle 
	& v_z &&\equiv \frac{d z}{d t}\\
	& &&=\underbrace{\underbrace{\frac{\partial z}{\partial s}\bigg\rvert_{tx}}_{\equiv h} \frac{d s}{dt}}_{\equiv v_s}
	+\underbrace{\frac{\partial z}{\partial x}\bigg\rvert_{ts} \underbrace{\frac{d x}{dt}}_{\equiv u}
	+\frac{\partial z}{\partial t}\bigg\rvert_{xs} \underbrace{\frac{d t}{dt}}_{=1}}_{=\frac{dz}{dt}\big\rvert_{s}}\\[4mm]
	& &&=\frac{\partial z}{\partial s}\bigg\rvert_{tx} \frac{d s}{dt}
	+\frac{d z}{d t}\bigg\rvert_{s} \\[4mm]
	& &&=\ \ h \frac{d s}{dt}\quad
	+\frac{d z}{d t}\bigg\rvert_{s}\\[4mm]
	& &&=
	\ \ v_s 
	\qquad+\underbrace{\frac{\partial z}{\partial t}\bigg\rvert_{xs}
	+u \frac{\partial z}{\partial x}\bigg\rvert_{ts}}
	_{\frac{d z}{d t}\big\rvert_{s}=v_{\Sigma,z}}
  \end{alignat}
  \label{eq_vertvelcomp}
\end{subequations}
where:
\begin{equation}
	\displaystyle
	h\equiv\frac{\partial z}{\partial s}\bigg\rvert_{tx} \ \ \text{and} \ \
	v_s\equiv h\frac{d s}{d t}
\end{equation}

In other words, the vertical velocity is the composition of $v_{\Sigma,z}$ (the vertical component of the velocity of the constant-$s$ surface as it moves), and $v_s$ (the velocity through this same surface). An important aspect of this computation is that $v_s$ remains a velocity along the vertical axes since no change of direction of the axes is made.

%%%%%%%%%%%%%%%%%%%%%%%%%%%%%%%%%%%%%%%%%%%%%%%%%%%%%%%%%%%%%%%%%%%%%%%%%%%%%%
%\subsubsection{Vertical velocities in $"\sigma"$-coordinates}
%%%%%%%%%%%%%%%%%%%%%%%%%%%%%%%%%%%%%%%%%%%%%%%%%%%%%%%%%%%%%%%%%%%%%%%%%%%%%
In the more restrictive case where $\sigma$-coordinates are used:
% In $\sigma$-coordinates:
\begin{equation}
 \displaystyle
 \sigma=\frac{z-\zeta}{H+\zeta}
\end{equation}
and as a consequence:
\begin{equation}
 \displaystyle
 v_z=w=\mathbf{u}_z.\mathbf{v}
=\frac{dz}{dt}=\underbrace{(H+\zeta)\frac{d\sigma}{dt}}_{\equiv v_{\sigma}}
 +(\sigma-1)\frac{dH}{dt}
 +\sigma\frac{d\zeta}{dt}
\end{equation}
where in $\sigma$-coordinates:
\begin{equation}
 \displaystyle
v_{\sigma}=(H+\zeta)\frac{d\sigma}{dt}
\end{equation}

%%%%%%%%%%%%%%%%%%%%%%%%%%%%
\subsubsection{Conservation of an extensive quantity in a free-surface ocean}
%%%%%%%%%%%%%%%%%%%%%%%%%%%

To develop diagnostics in a realistic free-surface ocean, the conservation or at least the evolution of a given property defined as $\mathcal{A}=\iiint_{\mathcal{V}(t)} \rho A d\tau$ where $A$ is a given specific\footnote{Specific: per unit mass of the flow.} property must be calculated in a region of the ocean with a varying upper surface. For a general time-dependent volume $\mathcal{V}(t)$ moving with a velocity $\mathbf{v}_{\Sigma}$, a formulation of the Reynolds transport theorem (or Leibnitz theorem) is given by \citet{truesdell_classical_1960} and reformulated in Appendix \ref{annexe_reynolds}:
\begin{subequations}
  \begin{alignat}{2}
  \displaystyle 
  &  \frac{d}{dt} \iiint_{\mathcal{V}(t)} \rho A d\tau && =
  \iiint_{\mathcal{V}(t)} \frac{D\rho A}{Dt}  d\boldsymbol{x}
  +\iiint_{\mathcal{V}(t)} \rho A \mathbf{\nabla}.\mathbf{v}_{\Sigma}\ d\mathbf{x}\\[4mm]
 & && =
  \iiint_{\mathcal{V}(t)} \frac{\partial \rho A}{\partial t}\bigg\rvert_{xz} d\tau
  %+ \varoiint_{\mathcal{V}(t)}\rho A \mathbf{v}.\mathbf{S}\\
  + \oiint_{\mathcal{S}(t)}\rho A   \mathbf{v}_{\Sigma}.\mathbf{n}_{\Sigma}dS_{\Sigma}
  \end{alignat}
\end{subequations}
with the material derivative $D/Dt$ here associated to $  \mathbf{v}_{\Sigma}$:
\begin{equation}
 \displaystyle
 \frac{D\bullet}{Dt}=\frac{\partial \bullet}{\partial t}\bigg\vert_{\boldsymbol{\xi}}
 +  \mathbf{v}_{\Sigma}.\mathbf{\nabla}_{t,\boldsymbol{\xi}}\bullet
\end{equation}
In oceanic configurations, $\mathcal{V}(t)$ can be chosen as $\mathcal{V}_s$, a sum of water columns. In the absence of shoaling processes, this volume's boundaries only move in the vertical direction with the elevation of the free-surface. It is then possible to express $\mathcal{A}$ in s-coordinates with the Jacobian of transformation equal to $\mathcal{J}=h$:
\begin{equation}
  \displaystyle 
 	\frac{d }{d t} \iiint_{\mathcal{V}_{s}} \rho A\ d\tau  =
 	\frac{d }{d t} \iiint_{\mathcal{V}_{s}} \rho A\ h ds dx_s dy_s
\end{equation}
In this case, with the definition of constant-$s$ surfaces, the boundaries of volume $\mathcal{V}_s$ are constant and the time-derivative can directly be applied through the integral as :
\begin{subequations}
  \begin{alignat}{2}
  \displaystyle 
   &\frac{d }{d t} \iiint_{\mathcal{V}_{s}} \rho A\ h ds dx_s dy_s && =
   \iiint_{\mathcal{V}_{s}} \frac{\partial h \rho A}{\partial t}\bigg \vert_s ds dx_s dy_s\\[4mm]
   & &&= \iiint_{\mathcal{V}_{s}} \rho h \frac{\partial A}{\partial t}\bigg \vert_s  ds dx_s dy_s + \iiint_{\mathcal{V}_{s}} A \frac{\partial \rho h}{\partial t}\bigg \vert_s ds dx_s dy_s
   \end{alignat}
\end{subequations}
This formulation is based on Reynolds transport theorem (see appendix \noparref{annexe_reynolds}).

To simplify notations and without any loss of generality, integrals should in the following be restricted to the $(x,z)$ vertical plane and subscript $s$ should be abandoned:
\begin{equation}
\displaystyle
\mathcal{A}=\int_x\int_{-1}^0\rho A\ hdx ds
\end{equation}
The conservation of mass in equation \ref{mass_s} and the formulation of evolution of $\rho$ in equation \ref{eq_diff_s} of appendix \ref{annexe_ocmod} can be used, leading to:
\begin{subequations}
  \begin{alignat}{2}
  \displaystyle 
   & \frac{d }{d t} \int_x\int_{-1}^0\rho A\ hds dx && = \int_x\int_{-1}^0 \frac{\partial A}{\partial t}\bigg \vert_s \rho h ds dx\\[4mm]
 & && \quad - \int_x \int_{-1}^0 A \frac{\partial \rho h u}{\partial x}\bigg\rvert_{ts} \ dx ds \\[4mm] 
 & && \quad - \int_x \int_{-1}^0 A \frac{\partial \rho v_s}{\partial s}\bigg\rvert_{tx} \ dx ds \\[4mm]
 & && \quad + \int_x \int_{-1}^0 A \frac{\partial}{\partial x} \bigg(h \kappa^h \frac{\partial \rho}{\partial x}\bigg\rvert_{ts}\bigg)_{ts} \ dx ds \\[4mm]
 & && \quad + \int_x \int_{-1}^0 A \frac{\partial}{\partial s} \bigg(\frac{\kappa^v}{h} \frac{\partial \rho}{\partial s}\bigg\rvert_{tx}\bigg)_{tx} \ dx ds 
   \end{alignat}
\end{subequations}
Various fluxes can be simplified in this expression:
\begin{subequations}
  \begin{alignat}{2}
&\frac{d }{d t}\int_x\int_{-1}^0 \rho A\ h ds dx =&&  \int_x \int_{-1}^0 \rho h \frac{\partial A}{\partial t}\bigg\rvert_{xs} \ dx ds\\[4mm]
 & && \quad - \int_x \int_{-1}^0 \frac{\partial \rho h A u}{\partial x}\bigg\rvert_{ts} \ dx ds
 + \int_x \int_{-1}^0\rho h u \frac{\partial A}{\partial x}\bigg\rvert_{ts} \ dx ds\\[4mm] 
 & && \quad - \int_x \int_{-1}^0 \frac{\partial \rho A v_s}{\partial s}\bigg\rvert_{tx} \ dx ds
 + \int_x \int_{-1}^0 \rho v_s \frac{\partial A}{\partial s}\bigg\rvert_{tx} \ dx ds \\[4mm]
 & && \quad + \int_x \int_{-1}^0 \frac{\partial}{\partial x} \bigg(h A \kappa^h \frac{\partial \rho}{\partial x}\bigg\rvert_{ts}\bigg)_{ts} \ dx ds 
 - \int_x \int_{-1}^0 h \kappa^h \frac{\partial A}{\partial x}\bigg\rvert_{ts} \frac{\partial \rho}{\partial x}\bigg\rvert_{ts} \ dx ds \\[4mm]
 & && \quad + \int_x \int_{-1}^0 \frac{\partial}{\partial s} \bigg( A \frac{\kappa^v}{h} \frac{\partial \rho}{\partial s}\bigg\rvert_{tx}\bigg)_{tx} \ dx ds 
 - \int_x \int_{-1}^0 \frac{\kappa^v}{h} \frac{\partial A}{\partial s}\bigg\rvert_{tx} \frac{\partial \rho}{\partial s}\bigg\rvert_{tx} \ dx ds 
  \end{alignat}
\end{subequations}
Integrating by parts the advective and diffusive flux integrals in both the vertical and horizontal directions finally leads to:
\begin{subequations}
\label{Aint}
  \begin{alignat}{2}
  \displaystyle 
  &\frac{d }{d t} \int_x \int_{-1}^0 \rho h A\ dx ds =&&
 %
 % Begin 5th line
 %
\quad  \int_x \int_{-1}^0 \rho h \frac{\partial A}{\partial t}\bigg\rvert_{ts} \ dx ds\\[4mm]
 & && \quad - \bigg[ \int_{-1}^0 \rho h A u \ ds\bigg]_{x}
 + \int_x \int_{-1}^0\rho h u \frac{\partial A}{\partial x}\bigg\rvert_{ts} \ dx ds\\[4mm] 
 & && \quad - \bigg[ \int_{x} \rho A v_s \ ds\bigg]_{0}^1
 + \int_x \int_{-1}^0 \rho v_s \frac{\partial A}{\partial s}\bigg\rvert_{ts} \ dx ds \\[4mm]
 & && \quad + \bigg[ \int_{-1}^0 h A \kappa^h \frac{\partial \rho}{\partial x}\bigg\rvert_{ts} \ ds \bigg]_{x}
 - \int_x \int_{-1}^0 h \kappa^h \frac{\partial A}{\partial x}\bigg\rvert_{ts} \frac{\partial \rho}{\partial x}\bigg\rvert_{ts} \ dx ds \\[4mm]
 & && \quad + \bigg[ \int_x A \frac{\kappa^v}{h} \frac{\partial \rho}{\partial s}\bigg\rvert_{tx} \ dx \bigg]_0^1
 - \int_x \int_{-1}^0 \frac{\kappa^v}{h} \frac{\partial A}{\partial s}\bigg\rvert_{tx} \frac{\partial \rho}{\partial s}\bigg\rvert_{tx} \ dx ds
   \end{alignat}
\end{subequations}
This latest relation can also be formulated as a function of $dA/dt$:
\begin{subequations}
  \begin{alignat}{2}
 %
 % Begin 6th line
 %
 &\frac{d }{d t} \int_x \int_{-1}^0 \rho h A\ dx ds &&= \quad  \int_x \int_{-1}^0 \rho h \frac{d A}{d t} \ dx ds\\[4mm]
 & && \quad - \bigg[ \int_{-1}^0 \rho h A u \ ds\bigg]_{x} - 0\\[4mm] 
 & && \quad + \bigg[ \int_{-1}^0 h A \kappa^h \frac{\partial \rho}{\partial x}\bigg\rvert_{ts} \ ds \bigg]_{x}
 - \int_x \int_{-1}^0 h \kappa^h \frac{\partial A}{\partial x}\bigg\rvert_{ts} \frac{\partial \rho}{\partial x}\bigg\rvert_{ts} \ dx ds \\[4mm]
 & && \quad + \bigg[ \int_x A \frac{\kappa^v}{h} \frac{\partial \rho}{\partial s}\bigg\rvert_{tx} \ dx \bigg]_0^1
 - \int_x \int_{-1}^0 \frac{\kappa^v}{h} \frac{\partial A}{\partial s}\bigg\rvert_{tx} \frac{\partial \rho}{\partial s}\bigg\rvert_{tx} \ dx ds 
  \end{alignat}
\end{subequations}
with $\bigg[ \int_{x} \rho A v_s \ ds\bigg]_{0}^1 = 0$ since $v_s$ is void at the bottom and top boundaries. 

%\subsection{S-coordinates}
%Integration over the water column whose surface is varying in time. 
%A key property is:
%\begin{equation}
% \displaystyle
% \frac{dh}{dt}(t,\mathbf{x},z)=\frac{d}{dt}\left( \frac{\partial z}{\partial s}\bigg\vert_{txy}\right)
% =\frac{\partial }{\partial s}\frac{dz}{dt}=\frac{\partial w}{\partial s}(t,\mathbf{x},z)
%\end{equation}


\subsection{Decomposition of potential energy balance in a free-surface ocean}
\label{section_PE_chap2}

The \textit{ocean model} proposed and detailed so far is self-consistent and can lead to the description of numerous dynamical processes over a wide range of space-time scales. To proceed, one needs to develop dedicated diagnostics and to focus on the mechanisms of particular interest. In this context, a central objective of the present study is to explicitly simulate large turbulent eddies in order to improve the representation of the transformation of water masses. In other words, the (implicit) modeling or the (explicit) simulation of \textit{mixing} is at stake.

The most rigorous approach of the \textit{mixing}-related issues has undoubtedly been proposed by \cite{lorenz_available_1955}. It is based on the concept of Background and Available gravitational Potential Energy (hereafter respectively BPE and APE) as a decomposition of the total gravitational Potential Energy (PE). The author defines first APE as the potential energy released when reorganizing adiabatically the flow so that it reaches its state of minimal potential energy. This minimal value corresponds to the BPE, that can also be expressed as the difference between the PE and the APE. The BPE compartment, thus defined, grows monotonically under diabatic mixing, offering a well-defined tool to evaluate this mixing.

Lorenz' approach, popularized by \cite{winters_available_1995}, remains challenging when applied in a realistic context, for instance over a limited area of a free-surface, varying-bathymetry ocean. It is even more challenging if applied in a numerical context with a s-coordinate model. A "limited area" raises indeed many questions on boundary fluxes and the validity of the state of minimal energy, whereas a free surface or a varying bathymetry both complicate the computation of this state in particular when s-coordinates are used.

As a consequence, localization and furthermore the accurate evaluation of hot spots of enhanced mixing in the context of the \textit{ocean model} presented so far remain a challenge. A general formulation of the evolution equation of both the PE and BPE is now proposed for a given volume made of water columns that expand vertically with the movements of the free-surface. The relations thus obtained are then used to characterize the mixing in various configurations of ocean regions in chapter \ref{chapBPE}.


\subsubsection{Reference state(s)}

The state of the stratification after the adiabatical reorganization of \citet{lorenz_available_1955} is called the Lorenz reference state. It consists in attributing a reference depth $z^*$ to each fluid parcel, the potential energy of this state, the BPE, can then be expressed as :
\begin{equation}
    E_b = \int_x \int_-1^0 g \rho z^* dx ds
\end{equation}
The evolution of BPE, through the evolution of the stratification profile $z^*$, reflects the adiabatic mixing processes that can then be quantified.

However, both \cite{saenz_estimating_2015} and Tailleux in his latest publications \citep{tailleux_local_2018} insist on the fact that this Lorenz reference state of absolute minimum potential energy is difficult to define and compute (due to numerical cost of sorting algorithm, non-linearity of the EOS, etc.). As a consequence, these studies are dedicated to the evaluation of the errors and discrepancies when simpler and easier-to-obtain reference states are defined (horizontal averaged, etc.). \cite{saenz_estimating_2015} additionally describes in detail an algorithm to compute Lorenz' reference state in an efficient way. They associate the computation of $Z^*(\rho)$ from the computation of the Level of Neutral Buoyancy (their LNB) and they separate it from the computation of Lorenz' reference state that is obtained first by inverting a non-linear relation. This algorithm is claimed to be much more efficient than classical adiabatic redistribution algorithms based on \citet{winters_available_1995}.

%Beyond the definition of a reference state and more precisely beyond the definition of a reference profile, several underlying ideas can be found:
%\begin{itemize}
%\setlength\itemsep{0pt}
%	\item The reference state is a basic (not necessarily resting but possibly balanced) state to which full dynamics is to be compared.
%	\item The Boussinesq assumption is itself based on the choice of a reference (constant) density (usually named here $\tilde{\rho}_{0}$) that is subtracted from the 3D total density. An hydrostatic pressure $p_{00}$ is defined by $\partial p_{00}/ \partial z=-\tilde{\rho}_{0} g$. Subtracting directly a vertically-varying reference density $\tilde{\rho}_{0}(z)$ leads to the less restrictive anelastic assumption. Interestingly enough, this Boussinesq reference state $(\tilde{\rho}_{0},\ p_{00})$ can be associated to the thermodynamic equilibrium state. \cite{tailleux_energetics_2009,tailleux_thermodynamicsdynamics_2012} associates this reference state to the \textit{internal dead energy} compartment (named $IE0$) concluding that "the viscous dissipation of kinetic energy (hereafter KE) and the diffusive dissipation of APE" increase the reference temperature $T_{00}$ (decreasing $\tilde{\rho}_{0}$) by increasing the IE0 (page 9 of the latter publication).
%	\item The reference profile (named here $\rho_0$) with both time and space (vertical) dependencies is then defined by $\rho(t,\mathbf{x})=\tilde{\rho}_{0}+\rho_0(t,z)+\rho'(t,\mathbf{x})$.  \cite{tailleux_thermodynamicsdynamics_2012} (still page 9) further claims that "turbulent mixing smooths out the vertical gradient of $\rho_0(t,z)$ after defining the exergy compartment of the internal energy. An hydrostatic pressure component $p_0$ can then be defined in association with $\rho_0$.
%	\item The reference state $\rho_0(t,z)$ is additionally used by \citet{andrews_note_1981} and \citet{tailleux_local_2018} to decompose the available potential energy into its elastic and potential compartments. This is done by defining two states. A transformation can be applied to a given fluid volume in such a way that its entropy, salinity and pressure $(\eta,S,p)$ themselves transformed (i) into $(\eta,S,\tilde{p}_0(t,z))$ (changing the reference pressure at $(t,z)$ while keeping $\eta$ and $s$ constant) and (ii) into $(\eta,S,p_0(z_r))$ (transferring the particulate to its "Level of Neutral Buoyancy" (LNB) given by $z=z_r$ while keeping $\eta$ and $S$ constant. 
%%\textit{Link to be studied with \cite{mcdougall_potential_2003}'s conservation of potential enthalpy}.
%\end{itemize}
It is crucial to know what state of the ocean a given reference state relates to in order to work within the BPE framework. 

%%%%%%%%%%%%%%%%%%%%%%%%%%%%%%%%%%%%%%%%%%%%%%%%%%%%%%%%%%%%%%%%%%%%%%%%%%%%%
\subsubsection{Balance of PE in s-coordinates}
%%%%%%%%%%%%%%%%%%%%%%%%%%%%%%%%%%%%%%%%%%%%%%%%%%%%%%%%%%%%%%%%%%%%%%%%%%%%%
Evolution integrals developed in the previous section \ref{subsection_scoord} can be formulated for the specific PE by setting $A=gz$ :
%\paragraph{Potential energy}
\begin{subequations}
  \begin{alignat}{2}
  \displaystyle 
 	&\frac{d E_p}{d t}  &&=g\int_x \int_{-1}^0 \frac{\partial \rho h z}{\partial t}\bigg\rvert_{s} dx ds 
  \end{alignat}
\end{subequations}
Replacing $A$ in expression \ref{Aint}, that uses the evolution equation for density \ref{eq_diff_cart} :
\begin{subequations}
  \begin{alignat}{2}
  \displaystyle 
 %
 % Begin 2nd line
 %
 &\frac{d E_p}{d t}   &&= \quad  g\int_x \int_{-1}^0 \rho h \frac{\partial z}{\partial t}\bigg\rvert_{xs} \ dx ds\\[4mm]
 %
 & && \quad - g\bigg[ \int_{-1}^0 \rho h z u \ ds\bigg]_{x}
 + g\int_x \int_{-1}^0\rho h u \frac{\partial z}{\partial x}\bigg\rvert_{ts} \ dx ds\\[4mm] 
 & && \quad - g\bigg[ \int_x \frac{\partial \rho z v_s}{\partial s}\bigg\rvert_{tx} \ dx\bigg]_0^1
 + g\int_x \int_{-1}^0 \rho v_s \frac{\partial z}{\partial s}\bigg\rvert_{tx} \ dx ds \\[4mm]
 & && \quad + g\bigg[ \int_{-1}^0 h z \kappa^h \frac{\partial \rho}{\partial x}\bigg\rvert_{ts} \ ds \bigg]_{x}
 - g\int_x \int_{-1}^0 h \kappa^h \frac{\partial z}{\partial x}\bigg\rvert_{ts} \frac{\partial \rho}{\partial x}\bigg\rvert_{ts} \ dx ds \\[4mm]
 & && \quad + g\bigg[ \int_x z \frac{\kappa^v}{h} \frac{\partial \rho}{\partial s}\bigg\rvert_{tx} \ dx \bigg]_0^1
 - g\int_x \int_{-1}^0 \frac{\kappa^v}{h} \frac{\partial z}{\partial s}\bigg\rvert_{tx} \frac{\partial \rho}{\partial s}\bigg\rvert_{tx} \ dx ds
  \end{alignat}
\end{subequations}
The advective flux of PE (fourth term at the RHS) can be integrated by part leading to:

\begin{subequations}
  \begin{alignat}{2}
  \displaystyle 
 	&\frac{d E_p}{d t}  &&= \quad  \underbrace{g\int_x \int_{-1}^0 \rho h \frac{\partial z}{\partial t}\bigg\rvert_{xs} \ dx ds}_{(1)}\\[4mm]
  %
 & && \quad - g\bigg[ \int_{-1}^0 \rho h z u \ ds\bigg]_{x}
 + \underbrace{g\int_x \int_{-1}^0\rho h u \frac{\partial z}{\partial x}\bigg\rvert_{ts} \ dx ds}_{(2)}\\[4mm] 
 & && \quad - \underbrace{g\bigg[ \int_x\rho z v_s \ dx\bigg]_0^1}_{=0}
 + \underbrace{g\int_x \int_{-1}^0 \rho h v_s \ dx ds}_{(3)} \\[4mm]
 & && \quad + g\bigg[ \int_{-1}^0 h z \kappa^h \frac{\partial \rho}{\partial x}\bigg\rvert_{ts} \ ds \bigg]_{x} 
 - g\int_x \int_{-1}^0 h \kappa^h \frac{\partial z}{\partial x}\bigg\rvert_{ts} \frac{\partial \rho}{\partial x}\bigg\rvert_{ts} \ dx ds \\[4mm]
 & && \quad + g\bigg[ \int_x z \frac{\kappa^v}{h} \frac{\partial \rho}{\partial s}\bigg\rvert_{tx} \ dx \bigg]_0^1
 - g\int_x \int_{-1}^0 \kappa^v \frac{\partial \rho}{\partial s}\bigg\rvert_{tx} \ dx ds 
  \end{alignat}
\end{subequations}
where the advection of PE at the surface and at the bottom vanishes. At the RHS, terms (1), (2) and (3) can be gathered to recover the buoyancy flux component $(\phi_z)$ of \citet{winters_available_1995}.
\begin{subequations}
  \begin{alignat}{2}
  \displaystyle 
 	&\frac{d E_p}{d t}  &&= \quad  \underbrace{\int_x \int_{-1}^0 \rho g h v_z \ dx ds}_{=(1)+(2)+(3)\equiv\phi_z}\\[4mm]
 %
 & && \quad - g\bigg[ \int_{-1}^0 \rho h z u \ ds\bigg]_{x}\\[4mm] 
 & && \quad + g\bigg[ \int_{-1}^0 h z \kappa^h \frac{\partial \rho}{\partial x}\bigg\rvert_{ts} \ ds \bigg]_{x}
 - g\int_x \int_{-1}^0 h \kappa^h \frac{\partial z}{\partial x}\bigg\rvert_{ts} \frac{\partial \rho}{\partial x}\bigg\rvert_{ts} \ dx ds \\[4mm]
 & && \quad + g\bigg[ \int_x z \frac{\kappa^v}{h} \frac{\partial \rho}{\partial s}\bigg\rvert_{tx} \ dx \bigg]_0^1
 - g\int_x \int_{-1}^0 \kappa^v \frac{\partial \rho}{\partial s}\bigg\rvert_{tx} \ dx ds
  \end{alignat}
\end{subequations}
%%%%%%%%%%%%%%%%%%%%%%%%%%%%%%%%%%%%%%%%%%%%%%%%%%%%%%%%%%%%%%%%%%%%%%%%%%%%%
%\paragraph{Background Potential energy}
%%%%%%%%%%%%%%%%%%%%%%%%%%%%%%%%%%%%%%%%%%%%%%%%%%%%%%%%%%%%%%%%%%%%%%%%%%%%%
%%%%%%%%%%%%%%%%%%%%%%%%%%%%%%%%%%%%%%%%%%%%%%%%%%%%%%%%%%%%%%%%%%%%%%%%%%%%%
\subsubsection{Balance of BPE in s-coordinates}
%%%%%%%%%%%%%%%%%%%%%%%%%%%%%%%%%%%%%%%%%%%%%%%%%%%%%%%%%%%%%%%%%%%%%%%%%%%%%
The same relation \ref{Aint} can alternatively be used for $A=gz^*$, the specific BPE, leading to:
\begin{subequations}
\label{eq_evolBPE0}
  \begin{alignat}{2}
  \displaystyle 
 	&\frac{d E_b}{d t} &&=
 	g\int_x \int_{-1}^0 \frac{\partial \rho h z^*}{\partial t}\bigg\rvert_{s} \ dx ds
  \end{alignat}
\end{subequations}
and then 	
\begin{subequations}
  \begin{alignat}{2}
  \displaystyle 
 %
 % Begin 2nd line
 %
 & \frac{d E_b}{d t} &&= \quad  {g\int_x \int_{-1}^0 \rho h \frac{\partial z^*}{\partial t}\bigg\rvert_{xs} \ dx ds}_{(1)}\\[4mm]
  %
 & && \quad - g\bigg[ \int_{-1}^0 \rho h z^* u \ ds\bigg]_{x}
 + {g\int_x \int_{-1}^0\rho h u \frac{\partial z^*}{\partial x}\bigg\rvert_{ts} \ dx ds}_{(2)}\\[4mm] 
 & && \quad - 0
 + {g\int_x \int_{-1}^0 \rho v_s \frac{\partial z^*}{\partial s}\bigg\rvert_{tx} \ dx ds}_{(3)} \\[4mm]
 & && \quad + g\bigg[ \int_{-1}^0 h z^* \kappa^h \frac{\partial \rho}{\partial x}\bigg\rvert_{ts} \ ds \bigg]_{x}
 - g\int_x \int_{-1}^0 h \kappa^h \underbrace{\frac{\partial z^*}{\partial x}\bigg\rvert_{ts} }_{=\frac{d z^*}{d\rho}\frac{\partial \rho}{\partial x}\big\rvert_{ts}\color{black}}\frac{\partial \rho}{\partial x}\bigg\rvert_{ts} \ dx ds \\[4mm]
 & && \quad + g\bigg[ \int_x z^* \frac{\kappa^v}{h} \frac{\partial \rho}{\partial s}\bigg\rvert_{tx} \ dx \bigg]_0^1
 - g\int_x \int_{-1}^0 \frac{\kappa^v}{h} \frac{\partial z^*}{\partial s}\bigg\rvert_{tx} \frac{\partial \rho}{\partial s}\bigg\rvert_{tx} \ dx ds
  \end{alignat}
\end{subequations}
The terms associated to the particulate derivative of $z^*$ can be gathered from terms (1), (2) and (3):
%\begin{subequations}
%  \begin{alignat}{2}
%  \displaystyle 
% 	&\frac{d E_b}{d t}  &&= \quad g\int_x \int_{-1}^0 \rho h \frac{\partial z^*}{\partial t}\bigg\rvert_{xs}
% +\rho v_s \frac{\partial z^*}{\partial s}\bigg\rvert_{tx} 
%+\rho h u \frac{\partial z^*}{\partial x}\bigg\rvert_{ts} \ dx ds \\[4mm]
% & && \quad - g\bigg[ \int_{-1}^0 \rho h z^* u \ ds\bigg]_{x}\\[4mm] 
% & && \quad - 0 \\[4mm]
% & && \quad + g\bigg[ \int_{-1}^0 h z^* \kappa^h \frac{\partial \rho}{\partial x}\bigg\rvert_{ts} \ ds \bigg]_{x}
% - g\int_x \int_{-1}^0 h \kappa^h \frac{d z^*}{d \rho} \frac{\partial \rho}{\partial x}\bigg\rvert_{ts}^2 \ dx ds \\[4mm]
% & && \quad + g\bigg[ \int_x z^* \frac{\kappa^v}{h} \frac{\partial \rho}{\partial s}\bigg\rvert_{tx} \ dx \bigg]_0^1
% - g\int_x \int_{-1}^0 \frac{\kappa^v}{h} \frac{d z^*}{d \rho} \frac{\partial \rho}{\partial s}\bigg\rvert_{tx}^2 \ dx ds
%  \end{alignat}
%\end{subequations}
%to explicitly express $dE_b/dt$ as a function of $dz^*/dt$:
\begin{subequations}
\label{eq_evolBPE}
  \begin{alignat}{2}
  \displaystyle 
 	&\frac{d E_b}{d t}  &&= \quad \underbrace{g\int_x \int_{-1}^0 \rho h \frac{d z^*}{d t}\ dx ds }_{(1)+(2)+(3)}\\[4mm]
 & && \quad - \underbrace{g\bigg[ \int_{-1}^0 \rho h z^* u \ ds\bigg]_{x}}_{\approx S_{adv}}\\[4mm] 
 & && \quad + \underbrace{g\bigg[ \int_{-1}^0 h z^* \kappa^h \frac{\partial \rho}{\partial x}\bigg\rvert_{ts} \ ds \bigg]_{x}}_{S_{diff}^{(h)}}
 \underbrace{- g\int_x \int_{-1}^0 h \kappa^h \frac{d z^*}{d \rho} \frac{\partial \rho}{\partial x}\bigg\rvert_{ts}^2 \ dx ds}_{\phi_d^{(h)}} \\[4mm]
 & && \quad + \underbrace{g\bigg[ \int_x z^* \frac{\kappa^v}{h} \frac{\partial \rho}{\partial s}\bigg\rvert_{tx} \ dx \bigg]_0^1}_{S_{diff}^{(v)}}
 \underbrace{- g\int_x \int_{-1}^0 \frac{\kappa^v}{h} \frac{d z^*}{d \rho} \frac{\partial \rho}{\partial s}\bigg\rvert_{tx}^2 \ dx ds}_{\phi_d^{(v)}}
  \end{alignat}
\end{subequations}
%Substituting $hdz^*/dt$ by the product of $dz^*/d\rho$ by $hd\rho/dt$ leads to\color{red}(en a besoin?)\color{black}:
%\begin{subequations}
%  \begin{alignat}{2}
%  \displaystyle 
% 	&\frac{d E_b}{d t}  &&= \quad \underbrace{g\int_x \int_{-1}^0  
% \rho  \frac{d z^*}{d\rho} \bigg[h \color{black}
% \frac{\partial \rho}{\partial t}\bigg\rvert_{xs} 
% +\rho v_s \frac{\partial \rho}{\partial s}\bigg\rvert_{tx} 
%+\rho h u \frac{\partial \rho}{\partial x}\bigg\rvert_{ts}
%\bigg] \ dx ds}_{(1)}\\[4mm]
% & && \quad - \underbrace{\bigg[ \int_{-1}^0 \rho g h z^* u \ ds\bigg]_{x}}_{\approx S_{adv}}\\[4mm] 
% & && \quad + \underbrace{g\bigg[ \int_{-1}^0 h z^* \kappa^h \frac{\partial \rho}{\partial x}\bigg\rvert_{ts} \ ds \bigg]_{x}}_{S_{diff}^{(h)}}
% - \underbrace{g\int_x \int_{-1}^0 h \kappa^h \frac{d z^*}{d \rho} \frac{\partial \rho}{\partial x}\bigg\rvert_{ts}^2 \ dx ds}_{\phi_d^{(h)}} \\[4mm]
% & && \quad +\underbrace{ g\bigg[ \int_x z^* \frac{\kappa^v}{h} \frac{\partial \rho}{\partial s}\bigg\rvert_{tx} \ dx \bigg]_0^1}_{S_{diff}^{(v)}}
% - \underbrace{g\int_x \int_{-1}^0 \frac{\kappa^v}{h} \frac{d z^*}{d \rho} \frac{\partial \rho}{\partial s}\bigg\rvert_{tx}^2 \ dx ds}_{\phi_d^{(v)}}
% %
% % Begin 5th line
% %
%% & &&= \quad g\int_x \int_{-1}^0 \rho h \frac{\partial z^*}{\partial t}\bigg\rvert_{xz}
% %+\rho v_z \frac{\partial z^*}{\partial z}\bigg\rvert_{tx} 
%%+\rho h u \frac{\partial z^*}{\partial x}\bigg\rvert_{tz} \ dx ds \\
% %& && \quad - \bigg[ \int_{-1}^0 \rho g h z^* u \ ds\bigg]_{x}\\ 
%% & && \quad + g\bigg[ \int_{-1}^0 h z^* \kappa^h \frac{\partial \rho}{\partial x}\bigg\rvert_{ts} \ ds \bigg]_{x}
%% - g\int_x \int_{-1}^0 h \kappa^h \frac{\partial z^*}{\partial \rho} \frac{\partial \rho}{\partial x}\bigg\rvert_{ts}^2 \ dx ds \\
%% & && \quad + g\bigg[ \int_x z^* \frac{\kappa^v}{h} \frac{\partial \rho}{\partial s}\bigg\rvert_{tx} \ dx \bigg]_0^1
%% - g\int_x \int_{-1}^0 \frac{\kappa^v}{h} \frac{\partial z^*}{\partial \rho} \frac{\partial \rho}{\partial s}\bigg\rvert_{tx}^2 \ dx ds 
%  \end{alignat}
%\end{subequations}
where $S_{adv}$ is the lateral (horizontal) advective flux of BPE, $S^{(v)}_{diff}$ and $S^{(h)}_{diff}$ are the vertical and horizontal diffusive fluxes, $\phi^{(v)}_d$ and $\phi^{(h)}_d$ are the diapycnal vertical and horizontal fluxes.

%%%%%%%%%%%%%%%%%%%%%%%%%%%%%%%%%%%%%%%%%%%%%%%%%%%%%%%%%%%%%%%%%%%%%%%%%%%%%
\subsubsection{Time variations of $z^*$}
%%%%%%%%%%%%%%%%%%%%%%%%%%%%%%%%%%%%%%%%%%%%%%%%%%%%%%%%%%%%%%%%%%%%%%%%%%%%%
The first RHS term of equation \ref{eq_evolBPE}, 

\begin{equation}
\label{RHS_BPE_1}
g\int_{x} \int_{-1}^0 \rho h \frac{d z^*}{d t}\ dx ds \\
\end{equation}

is akin to a vertical buoyancy-flux term for the $z^*$ reference state. It vanishes for a volume of integration with fixed-in-time boundaries (\cite{winters_available_1995}, \cite{huang_mixing_1998} and many others...). But it has no particular reason to vanish when movements of the free-surface are taken into account. This paragraph constitutes a proposition for its development in this case.

Going back to a more general notation of our integral volume $V_{\zeta}$ and following \citet{huang_mixing_1998}, $z^*$ can be expressed as a function of the volume $v(\rho)$ of fluid with a density larger than $\rho$ :
%En annexe de \citep{huang_mixing_1998}, on peut exprimer $z^*$ fonction de $v(\rho)$ le volume des parcelles de densité supérieures ou égales à $\rho$ :
\begin{equation}
v(\rho)=\iiint_{\zeta} H(\rho(\mathbf{x}',t)-\rho(\mathbf{x},t))d\mathbf{x}' %= \int_x\int_{-1}^0 H(\rho(x',s',t)-\rho(x,s,t))h ds'dx'
\end{equation}
where $H$ is the Heaviside function, equals to 0 for $\rho(\mathbf{x}',t)<\rho(\mathbf{x},t)$, $1/2$ for $\rho(\mathbf{x}',t)=\rho(\mathbf{x},t)$ , and 1 for $\rho(\mathbf{x}',t)>\rho(\mathbf{x},t)$.

In the reference profile, a given density is associated to a reference depth. This volume can then be expressed as $v(z^*)$, the volume of water between the bottom and the $z^*$ level in the water column :
\begin{equation}
v(z^*)=\int^{z^*}_{bottom} S(z')dz'
\end{equation}
with $S(z)$ the area at depth $z$.

%\textbf{\textit{Derivative of $z^*(t,\rho)$}}
%\begin{subequations}
%\begin{alignat}{2}
%&\frac{d z^*}{d t} &&= \frac{d z^*}{d v}\frac{d v}{d t}\\
%& && = \quad \bigg(\frac{d v}{d z^*}\bigg)^{-1}\bigg[\frac{\partial v}{\partial t}+u\frac{\partial v}{\partial x}+\frac{v_s}{h}\frac{\partial v}{\partial z}\bigg] \\
%%& && = \quad \frac{1}{A(z^*)}\bigg[\frac{\partial v}{\partial t}+u\frac{\partial v}{\partial x}+\frac{v_s}{h}\frac{\partial v}{\partial s}\bigg]\\
%& && = \quad \frac{1}{A(z^*)} \frac{d v}{d t}\\
%%& && \quad + u \bigg [ \int_{-1}^0 H(\rho(x,s',t)-\rho(x,s,t))h ds'\bigg]_x\\
%%& && \quad + \frac{v_s}{h} \bigg [ \int_x H(\rho(x',s,t)-\rho(x,s,t))h dx'\bigg]_0^1 \bigg]
%\end{alignat}
%\end{subequations}

Using the s-coordinate formulation of Reynolds transport theorem of equation \ref{eq_reyn_zeta} in the appendix's section \ref{annexe_reynolds}:
\begin{subequations}
  \begin{alignat}{2}
  \displaystyle
  & \frac{dz^*}{dt} &&=\bigg ( \frac{d \ v(z^*)}{dz^*} \bigg )^{-1} \frac{d \ v(\rho)}{dt}\\[4mm]
  & &&= \frac{1}{S(z^*(\rho))}\frac{d}{dt} \iiint_{V_{\zeta}} H\left(\rho(\mathbf{x}',t)-\rho(\mathbf{x},t)\right) d\mathbf{x}'\\[4mm]
  & && = \frac{1}{S(z^*(\rho))} \iiint_{V_{\zeta}} \frac{\partial}{\partial t} H\left(\rho(\mathbf{x}',t)-\rho(\mathbf{x},t)\right) d\mathbf{x}'\\[4mm]
  & && \quad  +\frac{1}{S(z^*(\rho))}\iint_{\mathcal{S}_{surf}} H\left(\rho(\mathbf{x}',t)-\rho(\mathbf{x},t)\right) \frac{\partial\zeta(x',t)}{\partial t} dS'
  \end{alignat}
\end{subequations}

%\begin{subequations}
%\begin{alignat}{2}
%& \frac{d Z^*}{d t} && = \frac{1}{S(Z(\rho))}\int_x\int_{-1}^0 H(\rho(x',s',t)-\rho(x,s,t))  \underbrace{\frac{\partial h}{\partial t}}_{= \ 0 \ sans \ mouvement \ surface \ libre ?}ds'dx'\\
%& && \quad + \underbrace { \int_x\int_{-1}^0 \frac{\partial }{\partial t} [H(\rho(x',s',t)-\rho(x,s,t))] \ h ds'dx'}_{\int_x \int_{-1}^0 \frac{1}{A} \rho h  ( \ ) dx ds \ = \ 0 ?} 
%\end{alignat}
%\end{subequations}
Where $S_{surf}$ is the free-surface.
As a consequence, the first RHS term expressed by \ref{RHS_BPE_1} can be rewritten as :
\begin{subequations}
\label{buoyancyzstar}
  \begin{alignat}{2}
  \displaystyle
  & \iiint_{\mathbf{V}_{\zeta}} \rho g \frac{dz^*}{dt} d\mathbf{x} &&=
\underbrace{ \iiint_{\mathbf{V}_{\zeta}} \frac{\rho g}{S(z^*(\rho))} \iiint_{V_{\zeta}} \frac{\partial}{\partial t} H\left(\rho(\mathbf{x}',t)-\rho(\mathbf{x},t)\right) d\mathbf{x}' d\mathbf{x}}_{=0}\\[4mm]
  & &&+\iiint_{\mathbf{V}_{\zeta}} \frac{\rho g}{S(z^*(\rho))}\iint_{\mathcal{S}_{surf}} H\left(\rho(\mathbf{x}',t)-\rho(\mathbf{x},t)\right) \frac{\partial\zeta(x',y',t)}{\partial t} dS' d\mathbf{x}
  \end{alignat}
\end{subequations}
The first term at the RHS vanishes as shown for instance by \cite{huang_mixing_1998}. The second term, however, will be \textit{a priori} non-zero for any case of with moving free-surface.

Note that with the definition of the Heaviside function, a way to interpret this term is that the contribution of the evolution of the free-surface elevation is only taken into account in the evolution of BPE for fluid parcels as light as the ones at the surface (ie, the parcels at the free-surface themselves), or lighter than them. Those parcels are placed near the top of the reference state's profile, so that $S(z^*(\rho))$ is in most cases the area at the level of the free-surface (but not the area of the deformed free-surface itself).


 
 \section{CROCO: a numerical implementation of the non-hydrostatic, compressible, free-surface \textit{ocean model}}
 \label{section_croco}
 
%----------------------------------------------------------------------------  
\subsection{Numerical implementation of the \textit{ocean model}}
%----------------------------------------------------------------------------
Ocean models whether dedicated to global, regional or even coastal scales are traditionally based on the Boussinesq, hydrostatic assumption \citep{griffies_elements_2012,shchepetkin_regional_2005}. The present study is a step toward the explicit simulation of at least the largest turbulent eddies in a realistic context and, as a consequence, a non-hydrostatic numerical approach is required. \cite{Auclair2018} concluded that an efficient non-hydrostatic, free-surface, mode-splitting numerical model of the ocean could be designed relaxing also the Boussinesq approximation. Doing so, the authors chose to work with local equations and they do not solve for a 3D Poisson equation to diagnose total pressure. They consequently follow the choices made in meso-scale atmospheric modeling by \cite{skamarock_prototypes_2001}. The compressible (non-Boussinesq) approach is original in ocean modeling and in particular in free-surface, ocean modeling. Indeed \cite{marshall_finite-volume_1997} or \cite{auclair_non-hydrostatic_2011} chose to retain the Boussinesq assumption. A consequence of \cite{Auclair2018}'s choice is that the complete \textit{ocean model} presented in \S\ref{section_prim_eq} can be solved numerically.

The computing cost of such a non-hydrostatic, compressible, free-surface approach can quickly become prohibitive especially because the explicit modeling of fine scales requires high-resolution grids. Following the conclusions of the COMODO french community\footnote{COMODO gathered the french ocean modeling community. It was sponsored by the French ANR eponymous project (2011-2016).}, the compressible and free-surface algorithm developed by \cite{Auclair2018} has been implemented in the ROMS-AGRIF branch of the ROMS ocean models \citep{shchepetkin_regional_2005}. This choice was justified by the great efficiency of Shchepelkin's time-splitting and time-stepping and more generally by the experience accumulated in ROMS community during the last decade.

The simulations of the strait of Gibraltar presented in chapters \ref{chapGBR2D} and \ref{chapGBR3D} were the very first realistic implementation of the non-hydrostatic, compressible, free-surface kernel of CROCO \citep{hilt_2020}.
\color{black}
%----------------------------------------------------------------------------  
\subsection{Time-splitting}
%----------------------------------------------------------------------------
\subsubsection{Dynamical time-scales}
Numerical constraints can conveniently be enumerated in terms of time-scales of dynamical "transfers" of tracer, pressure or velocity anomalies in the ocean. Advection, diffusion or radiation by gravity or acoustic waves are examples of such transfers. For a given length-scale (such as a model grid scale), maximum characteristic velocities can lead to an order of magnitude of the most restrictive time-scales for each type of "transfer".\\
Pressure and density anomalies are conveniently defined with respect to the hydrostatic rest state leading to the pressure decomposition:
\begin{equation}
	\displaystyle
	\label{decompoP_0}
	p(\mathbf{x},t)=p_h(\mathbf{x},t)+\delta p(\mathbf{x},t)
\end{equation}
with $p_h(\mathbf{x},t)$ the hydrostatic pressure component and $\delta p(\mathbf{x},t)$ an anomaly. The former is defined by $\partial_z p_h=-\rho_h(\mathbf{x},t) g$ where $\rho_h(\mathbf{x},t)$ can be chosen as the slowly-varying, statically-stable field of density. Based on this pressure decomposition, a first-order Taylor expansion of the density field can be carried out:
\begin{equation}
  \displaystyle 
	\label{decompor_0}
  \rho(T,S,p)=\rho_{\theta S}(T,S,p_0)+\frac{p_h+\delta p-p_0}{c_s^2}+\mathcal{O}(\delta p^2)
\end{equation}
for a reference, slow component of pressure $p_0$ which is most often chosen different from the hydrostatic pressure in numerical models.

Numerical constraints associated to the various transfers of anomalies can basically be classified into three categories depending if they are associated to compressibility (acoustic waves..), surface-induced processes (surface gravity waves...) or internal-ocean (incompressible) processes (internal gravity waves, advection, diffusion, buoyancy-induced processes...). Orders of magnitude of maximum velocities in a deep ocean of each category are respectively given by $v[\delta p]\approx \mathcal{O}(1500\ m/s)$, $v[p_\zeta]\approx\sqrt{g H}\approx \mathcal{O}(100\ m/s)$ and $v[p_{int},\ ...]\approx \mathcal{O}(1\ m/s)$ leading to at least two spectral gaps in terms of velocities in the ocean:
\begin{equation}
	\displaystyle
	\label{velocityscales}
	v[p_{int},\ ...] \ll v[p_\zeta] \ll v[\delta p]
\end{equation} 
This hierarchy of velocity (and thus time) scales and the associated gaps constitute the basis to develop time-splitting approaches for numerical models of the ocean.
Under free-surface, Boussinesq and hydrostatic assumptions, the time-splitting procedure implemented in ROMS model \citep{shchepetkin_regional_2005} filters for instance acoustic and non-hydrostatic processes and takes advantage of the gap $v[p_{int},\ ...] \ll v[p_\zeta]$. It can be formulated as a decomposition of the pressure between a 2D surface-induced pressure-component (named external or barotropic-like component) $\bar{p}_h(\mathbf{x},t)$ and a 3D density-induced (internal or baroclinic-like) pressure-component $p_h'(\mathbf{x},t)$. 

The time-splitting approach for a more general free-surface, non-hydrostatic and compressible ocean can also be based on \ref{velocityscales}. The procedure is yet different from that used for hydrostatic ocean models. In the latter, coupling is based on the separation of the velocity field between a "barotropic-like", depth-averaged component and a "baroclinic-like" anomaly. The faster, surface-induced component of the pressure force is integrated with a small time-step and after each integration sequence of the external mode, the depth-averaged component of the internal-mode velocity is forced to fit to the external-mode, depth-averaged velocity. Separating the "fast" and "slow" components of momentum in a compressible model to integrate them separately is not that simple and more importantly, it is not even necessary. The time-splitting procedure proposed in CROCO compressible kernel is indeed based on the splitting of the terms on the RHS of the prognostic and diagnostic equations of the ocean model. Two coupled models (hereafter called the slow and fast numerical kernels) are then integrated in turn. The slow (respectively fast) kernel is advanced with a larger (smaller) time-step computing explicitly slowly-varying (rapidly-varying) terms at the RHS and implicitly the remaining terms. A time-filtering procedure is implemented to force both the slow and fast mode in a similar way to \citet{shchepetkin_regional_2005}.

\subsubsection{Pressure and density decomposition}
The splitting of the processes based on the magnitude of their time-scale relies essentially on a decomposition of the pressure and density fields. Following \cite{auclair_modied_2021}, the pressure decomposition \ref{decompoP_0} can be further developed for a free-surface ocean:
\begin{subequations}
  \begin{alignat}{2}
  % Pressure decomposition
  \displaystyle 
 \label{decompoP_fa}
  &p(\mathbf{x},t) &&= 
  \underbrace{p_{atm}
  (\mathbf{x}_{\scriptscriptstyle H},t)
  +g\int_z^{\zeta}\rho_{h}(\mathbf{x}_{\scriptscriptstyle H},z',t)\ dz'}_{p_h(\mathbf{x},t)}
  +\delta p(\mathbf{x},t)\\[3mm]
  \label{decompoP_f}
  & &&= \underbrace{
  \underbrace{\rho_0 g\left(\zeta(\mathbf{x}_{\scriptscriptstyle H},t)-z\right)}_{\bar{p}_h(\mathbf{x},t)}
  +\underbrace{g\int_z^{\zeta}{\left(\rho_{h}(\mathbf{x}_{\scriptscriptstyle H},z',t)-\rho_0\right)\ dz'}}
  _{p_h'(\mathbf{x},t)}}_{p_h(\mathbf{x},t)}
  +\delta p(\mathbf{x},t)
  \end{alignat}
\end{subequations}
where $\rho_0$ is a constant reference density. 
The Taylor expansion of density with respect to total pressure \ref{decompor_0} leads then to:
\begin{subequations}
  \begin{alignat}{2}
  % Pressure decomposition
  \displaystyle 
  % Density decomposition
  &\rho(\mathbf{x},t) &&=\rho_{\theta S}(\mathbf{x},t)
  +\underbrace{\frac{1}{c_s^{2}}\left(p_h(\mathbf{x},t)+\delta p(\mathbf{x},t)-p_0(\mathbf{x},t)\right)}_{\left.\partial \rho / \partial p\right|_{T,S}\ (p(\mathbf{x},t)-p_0(\mathbf{x},t))} 
   +\, \mathrm{O}(p^2) \\[3mm]
  \label{decompor_f0}  
  & &&\approx\underbrace{\rho_h(\mathbf{x},t)+\rho_{nh}(\mathbf{x},t)
  +\frac{1}{c_s^{2}}\left(p_h(\mathbf{x},t)-p_0(\mathbf{x},t)\right)}_{\rho_{s}(\mathbf{x},t)}
  +\underbrace{\frac{\delta p(\mathbf{x},t)}{c_s^{2}}}_{\rho_f(\mathbf{x},t)}
  \end{alignat}
\end{subequations}
\noindent with $\partial p / \partial \rho|_\eta = c_s^2$ at constant entropy $\eta$, $\rho_{\theta S}=\rho_{eos}(\theta,\ S,\ p_0)$ and $\rho_{nh}=\rho_{\theta S}-\rho_h$. This decomposition of the pressure and density fields clearly demonstrate, if necessary, the inextricable relationships between compressibility and hydrostaticity assumptions. 

 %----------------------------------------------------------------------------  
 \subsubsection{Slow vs fast components}
 %----------------------------------------------------------------------------
Based on the decomposition of the pressure and density fields (\noparref{decompoP_f}, \noparref{decompor_f0}), the terms at the RHS of the momentum equations can be splitted in two categories depending on the time-scales they are associated with: 
\begin{subequations}
\label{momsf}
   \begin{alignat}{2}
   \displaystyle
   %%%%%%%%%%%%%%%%%%%%%%%%%%%%%%%%%%%%%%%%%%%%%%
   % Momentum
   %%%%%%%%%%%%%%%%%%%%%%%%%%%%%%%%%%%%%%%%%%%%%%
   &\partial_t\rho\mathbf{v} &&= 
   \underbrace{-\mathbf{\nabla}.\left(\rho\mathbf{v}\otimes\mathbf{v}\right)
   %-2\rho\mathbf{\Omega}\wedge\mathbf{v}
   -\rho f\mathbf{u_z}\wedge\mathbf{v}
   -\mathbf\nabla(\int\limits_z^{\zeta}{(\rho_{s}-\rho_0)g\ dz'})
   +\mu\Delta\mathbf{v}}_{\mathbf{\Lambda}_{s}}\\
   & && \quad \underbrace{-\rho_0 g\mathbf\nabla\zeta
   -\mathbf\nabla{\delta p}
   -\rho f'\mathbf{u_y}\wedge\mathbf{v}
   +\rho\mathbf{g}
   +\mu_2\mathbf{\nabla}(\mathbf{\nabla}.\mathbf{v})}_{\mathbf{\Lambda}_{f}}
   \end{alignat}
\end{subequations}
Note that the Coriolis pseudo-force is itself splitted: the traditional component (with $f=2\Omega sin(\phi)$, $\mathbf{u}_z$ the vertical unit vector in Cartesian coordinates and $\phi$ the latitude) is integrated with the slow kernel whereas the non-traditional component (with $f'=2\Omega cos(\phi)$ and $\mathbf{u}_y$ the south-north horizontal unit vector in Cartesian coordinates). This latter component can indeed be associated with horizontal-axis rolls and is integrated with the fast kernel. The nonlinear advective terms are integrated with the slow kernel, i.e. a priori with a larger time-step and thus at a lower cost. Diffusion terms associated to dynamical (respectively bulk) viscosity are integrated with the slow (fast) kernel. The momentum equation \ref{momsf} can thus be rewritten in a compact form as:
\begin{subequations}
\begin{alignat}{3}
 \displaystyle
 &\partial_t\rho h_s\mathbf{v}_s   &&=\quad\Lambda_s  &&+\ll\Lambda_f\gg\\[3mm]
 &\partial_t\rho h_f\mathbf{v}_f &&=\ [[\Lambda_s]]   &&+\quad\Lambda_f
\end{alignat}
\end{subequations}
This splitting conserves basically the formulation of the horizontal momentum equations proposed in \cite{shchepetkin_regional_2005}, the length-scales of the processes and the fast-mode forcing are yet obviously different but the filtering procedure $\ll.\gg$ is the "flat" filter proposed by \cite{shchepetkin_regional_2005}. $[[.]]$ notation indicates the extrapolation in time of the slow-kernel terms to be used at the fast-kernel RHS (see \S \noparref{TimeSplit}). 

%%%%%%%%%%%%%%%%%%%%%%%%%%%%%%%%%%%%%%%%%%%%%%%%%%%%%%%%%%%%%%%%%%%%%%%%%%%%%
\subsection{Time-stepping}
%%%%%%%%%%%%%%%%%%%%%%%%%%%%%%%%%%%%%%%%%%%%%%%%%%%%%%%%%%%%%%%%%%%%%%%%%%%%%
The time-splitting and time-stepping proposed in the following build both on \cite{shchepetkin_regional_2005} and on \cite{Auclair2018}.   \cite{shchepetkin_regional_2005}'s LFAM3\footnote{Leap-Frog Adams-Moulton 3 steps}, predictor-corrector time-stepping is indeed implemented in the slow kernel while a Forward-Backward (FB) scheme is used to integrate the fast-mode. The introduction of a compressible, non-hydrotatic kernel is taken from \cite{Auclair2018}.

Figure \ref{ModelTS} gives a schematic representation of the predictor-corrector implementation of the slow and fast kernels based on ROMS baroytropic/baroclinic time-splitting and both the time-splitting and the various time-stepping are summarized in Equations \ref{TimeSplit}.
\begin{figure}[!h]
	\centering		
	\begin{subfigure}{1.0\linewidth}
		\includegraphics[width=1\linewidth]{CHAP2/Model_TS.png}
		\caption{}
	\end{subfigure}
\caption{ \textit{time-splitting and time-stepping of CROCO model with its non-hydrostatic, compressible (NBQ) kernel. Yellow (blue) background color: slow (fast) kernel. }}
	\label{ModelTS}
\end{figure}
%
\begin{table}
\begin{subequations}
\label{TimeSplit}
\begin{alignat}{3}
 \displaystyle
 %%%%%%%%%%%%%%%%%%%%%%%%%%%%%%%%%%%%%%%%%%%%%%%%%%%%%%%%%%%%%
 &\nonumber \textbf{I.a Time-interpolation: } t_s-\Delta t_s/2\\[0mm]
 %%%%%%%%%%%%%%%%%%%%%%%%%%%%%%%%%%%%%%%%%%%%%%%%%%%%%%%%%%%%%
 \label{TimeSplitIa1}
 &\enspace[\Theta] ^{n-\frac{1}{2}}=\alpha_{n-1}\Theta_s^{n-1}
 +\alpha_{n}\Theta^{n}\\[3mm]
 %%%%%%%%%%%%%%%%%%%%%%%%%%%%%%%%%%%%%%%%%%%%%%%%%%%%%%%%%%%%%
 &\nonumber \textbf{I.b Predictor step: } t_s+\Delta t_s/2\\[0mm]
 %%%%%%%%%%%%%%%%%%%%%%%%%%%%%%%%%%%%%%%%%%%%%%%%%%%%%%%%%%%%%
 \label{TimeSplitIb1}
 &\enspace\rho h_s\mathbf{v}_s^{n+\frac{1}{2}}=
 \rho h_s\mathbf{v}_s^{n-\frac{1}{2}}
 +\Delta t_s\left(\Lambda_{s,v}^{n}+<\Lambda_{f,v}>^n\right)\\[3mm]
 %
 \label{TimeSplitIb2}
 &\enspace\rho h_s(\theta_s,\ S_s)^{n+\frac{1}{2}}=
 \rho h_s(\theta_s,\ S_s)^{n-\frac{1}{2}}
 +\Delta t_s\Lambda_{s,(\theta,S)}^{n}\\[3mm]
 %
 \label{TimeSplitIb3}
 &\enspace\rho_s^{n+\frac{1}{2}}=\rho_{eos}\left(\theta_s^{n+\frac{1}{2}},\ S_s^{n+\frac{1}{2}},\ z_s^{n+\frac{1}{2}}\right)\\[3mm]
 %
 \label{TimeSplitIb4}
 &\enspace\partial_s\rho\omega_s^{n+\frac{1}{2}}=-\partial_t\rho h_s^{n+\frac{1}{2}}
 +\mathbf{\nabla}\cdot\rho h_s \mathbf{u}_s^{n+\frac{1}{2}}\\[3mm]
 %%%%%%%%%%%%%%%%%%%%%%%%%%%%%%%%%%%%%%%%%%%%%%%%%%%%%%%%%%%%%
 &\nonumber \textbf{I.c AB3-extrapolation: } t_s+\Delta t_s/2\\[0mm]
 %%%%%%%%%%%%%%%%%%%%%%%%%%%%%%%%%%%%%%%%%%%%%%%%%%%%%%%%%%%%%
 \label{TimeSplitIc1}
 &\enspace[[\Psi_s]]^{n+\frac{1}{2}}=
  \beta_{n-2}\Psi_s^{n-2}
 +\beta_{n-1}\Psi_s^{n-1}
 +\beta_{n}\Psi_s^{n}\\[2mm]
 %%%%%%%%%%%%%%%%%%%%%%%%%%%%%%%%%%%%%%%%%%%%%%%%%%%%%%%%%%%%%
 &\nonumber \textbf{II. Fast-mode steps: } t_f\in(t_s,\ t_s+\Delta t_s] \textit{ or } m\in[0,\ N_f)_\mathcal{N}\\[2mm]
 %%%%%%%%%%%%%%%%%%%%%%%%%%%%%%%%%%%%%%%%%%%%%%%%%%%%%%%%%%%%%
 \label{TimeSplitIIa}
 &\enspace\zeta_f^{m+1}=\zeta_f^{m}+\Delta t_f\left(
  w_{surf}^{m}-\mathbf{u}_{surf}^{m}.\mathbf{\nabla}\zeta^{m}\right)\\[2mm]
 %
 \label{TimeSplitIIb}
 &\enspace\rho h u_f^{m+1}=
 \rho h u_f^{m}
 +\Delta t_f\left(
  [[\Lambda_{s,u}]]^{n+\frac{1}{2}}
 -[[\overline{\Lambda_{s,u}}]]^{n+\frac{1}{2}}
 +\Lambda_{f,u}^{m}
% +\overline{\overline{\Lambda_{f,u}}}^{\ m}
 +\overline{\overline{\Lambda_{f,u}}}^{\ m}
 +\overline{\overline{\Lambda_{f,-\mathbf{\nabla}\zeta}}}^{\ m+1}
 \right)\\[2mm]
 %
 \label{TimeSplitIIc}
 &\enspace\overline{\overline{\rho h U}}_f^{\ m+1}=
 \overline{\overline{\rho h U}}_f^{\ m}
 +\Delta t_f\left(
 %[[\overline{\Lambda_{s,u}}]]^{n+\frac{1}{2}}+
 \overline{\Lambda_{f,u}^{m}}
 +\overline{\overline{\Lambda_{f,u}}}^{\ m}
 +\overline{\overline{\Lambda_{f,-\mathbf{\nabla}\zeta}}}^{\ m+1}
 \right)\\[0mm]
 %
 \label{TimeSplitIId}
 &\enspace\rho h w_f^{m+1}=
 \rho h w_f^{m}
 +\Delta t_f\left([[\Lambda_{s,w}]]^{n+\frac{1}{2}}
 +\Lambda_{f,w}^{m+1*}\right)\\[2mm]
 %
 \label{TimeSplitIIe}
 &\enspace\rho h_f^{m+1}=\rho h_f^{m}
 -\Delta t_f\left(
 [[\partial_t\rho h_s]]^{n+\frac{1}{2}}
 +\mathbf{\nabla}\cdot\{\rho h \mathbf{v}\}_f^{m+1}
 \right)\\[0mm]
 %
 \label{TimeSplitIIh}
 &\enspace m=N_f-1:\ \bar{\rho}\zeta_s^{n+1}
 =\bar{\rho}(H+\zeta_f)^{m}
 -\bar{\rho}H_s^{m+1}
 -\Delta t_f\mathbf{\nabla}\cdot\overline{\overline{\rho h\mathbf{u}}}^{\ m+1}\\[2mm]
 %
 \label{TimeSplitIIg}
 &\enspace \textit{Update\ grid:}\ \rho h_f^{m+1},\ z_f^{m+1}\\[2mm]
 %
 %%%%%%%%%%%%%%%%%%%%%%%%%%%%%%%%%%%%%%%%%%%%%%%%%%%%%%%%%%%%%
 &\nonumber \textbf{III.a Filtering: } t_s+\Delta t_s\ \textit{and}\ t_s+\Delta t_s/2\\[0mm]
 %%%%%%%%%%%%%%%%%%%%%%%%%%%%%%%%%%%%%%%%%%%%%%%%%%%%%%%%%%%%%
 \label{TimeSplitIIIa1}
 &\enspace<\Phi_f>^{n+1}=\Phi_f^{m=n+1}\\[0mm]
 \label{TimeSplitIIIa2}
 &\enspace\ll\Phi_f\gg^{n+\frac{1}{2}}=\frac{1}{N_f}\sum_{m=1}^{N_f}\Phi_f^{m}\\[2mm]
 %%%%%%%%%%%%%%%%%%%%%%%%%%%%%%%%%%%%%%%%%%%%%%%%%%%%%%%%%%%%%
 &\nonumber \textbf{III.b Corrector step: } t_s+\Delta t_s\\[0mm]
 %%%%%%%%%%%%%%%%%%%%%%%%%%%%%%%%%%%%%%%%%%%%%%%%%%%%%%%%%%%%%
 %
 \label{TimeSplitIIIb1}
 &\enspace\rho h_s\mathbf{v}_s^{n+1}=
 \rho h_s\mathbf{v}_s^{n}
 +\Delta t_s\left(\Lambda_s^{n+\frac{1}{2}*}
 +\ll\Lambda_f\gg^{n+\frac{1}{2}}\right)\\[0mm]
 %
 \label{TimeSplitIIIb2}
 &\enspace\partial_s\rho\omega_s^{n+1}=
 -\partial_{t\ }\rho h_s^{n+1}
 +\mathbf{\nabla}\cdot\rho h_s \mathbf{u}_s^{\ n+1}
 -\overline{\mathbf{\nabla}\cdot\rho h_s \mathbf{u}_s}^{\ n+1}
 -\overline{<\mathbf{\nabla}\cdot\rho h_s \mathbf{u}_s>}^{\ n+1}\\[2mm]
 %
 \label{TimeSplitIIIb3}
 &\enspace\rho h_s(\theta_s,\ S_s)^{n+1}=
 \rho h_s(\theta_s,\ S_s)^{n}
 +\Delta t_s\Lambda_{s,(\theta,S)}^{n+\frac{1}{2}*}\\[0mm]
 %
 \label{TimeSplitIIIb4}
 &\enspace\rho_s^{n+1}=\rho_{eos}\left(\theta_s^{n+1},\ S_s^{n+1},\ z_s^{n+1}\right)\\[0mm]
 %
 \label{TimeSplitIIIb5}
 &\enspace\rho h_s\mathbf{u}_s^{n+1}=\rho h_s\mathbf{u}_s^{n+1}
 -\overline{\rho h_s\mathbf{u}_s}^{\ n+1}
 +\overline{\rho h\mathbf{u}_f}^{\ m=N_f-1}
 %%%%%%%%%%%%%%%%%%%%%%%%%%%%%%%%%%%%%%%%%%%%%%%%%%%%%%%%%%%%%
\end{alignat}
\end{subequations}
\end{table}
Predictor (I), fast-kernel Forward-Backward (II) and Corrector (III) steps are shown in horizontal color bands (yellow for the slow kernel, blue for the fast kernel) on figure \ref{ModelTS}. After time-interpolating slow-kernel variables to time-step $t_s-\Delta t_s/2$ (step I.a, notation $[.]$), the slow kernel is advanced from $t_s-\Delta t_s/2$ to $t_s+\Delta t_s/2$ with a centered, leap-frog-like, time-stepping (step I.b). Then, to prepare the integration of the fast kernel, the slow-kernel RHS is extrapolated to $t_s+\Delta t_s/2$ based on an AB3 scheme using its previous evaluations at $t_s-2\Delta t_s$, $t_s-\Delta t_s$ and $t_s$ (step I.c, notation $[[.]]$). The fast kernel can in turn be advanced from $t_s$ to $t_s+\Delta t_s$ using a forward-backward like time-stepping (step II) with time-step $\Delta t_f$ satisfying  $N_f=\Delta t_s/\Delta t_f\in\mathcal{N}$. The vertical grid is updated at each fast time-step \ref{TimeSplitIIg} but slow-kernel components of the RHS remain constant during the fast-kernel integration. At the last fast time-step, surface elevation displacement for the slow kernel can be recomputed to ensure perfect numerical coherence between the surface kinematic relation and depth-integrated mass conservation (\noparref{TimeSplitIIa} and \noparref{TimeSplitIIh}).\\
Further numerical details such as the values of the interpolation $(\alpha_n)$ or extrapolation $(\beta_n)$ coefficients, the expressions of the slow-kernel RHS terms $(\Lambda_s)$, the expressions of the fast-kernel surface-related pressure force terms $(\Lambda_{f,-\nabla\zeta})$,  the fast-kernel RHS remaining terms $(\Lambda_{f})$ or the implicit fast and slow-kernel RHS terms (indicated by an asterisk symbol) can be found in CROCO dedicated manuals and publications.\\  
%The barotropic-like, depth-independent component is also integrated with the same time-step $\Delta t_f$ with a forward-backward scheme as in \cite{shchepetkin_regional_2005}. 
A major difference with the hydrostatic time-splitting is that the surface elevation displacement is given by the kinematic condition \ref{TimeSplitIIa} and not by the depth-integral of the mass conservation equation. Once the fast-kernel RHS and variables have been filtered both at $t_s+\Delta t_s$ and $t_s+\Delta t_s/2$ (step III.a, notations $<.>$ and $\ll.\gg$), the slow kernel is finally advanced from $t$ to  $t_s+\Delta t_s$ during the leap-frog-like Corrector step (III.b).

Note that the 2D depth-averaged, barotropic-like, horizontal momentum equations \ref{TimeSplitIIc} are advanced in the same way as in \cite{shchepetkin_regional_2005}. The result of this 2D integration is indeed used to correct both the horizontal momentum itself and the RHS of the horizontal momentum equation at Corrector step. It can also be used to require a perfect coherence of the surface elevation displacement and the depth-average transport (at machine precision) during the slow-mode integration. 


%%%%%%%%%%%%%%%%%%%%%%%%%%%%%%%%%%%%%%%%%%%%%%%%%%%%%%%%%%%%%%%%%%%%%%%%%%%
%%%%%%%%%%%%%%%%%%%%%%%%%%%%%%%%%%%%%%%%%%%%%%%%%%%%%%%%%%%%%%%%%%%%%%%%%%%
%%%%%%%%%%%%%%%%%%%%%%%%%%%%%%%%%%%%%%%%%%%%%%%%%%%%%%%%%%%%%%%%%%%%%%%%%%%
\section{Appendices to the \textit{ocean model}}
\label{annexe_ocmod}
%%%%%%%%%%%%%%%%%%%%%%%%%%%%%%%%%%%%%%%%%%%%%%%%%%%%%%%%%%%%%%%%%%%%%%%%%%%
%%%%%%%%%%%%%%%%%%%%%%%%%%%%%%%%%%%%%%%%%%%%%%%%%%%%%%%%%%%%%%%%%%%%%%%%%%%
%%%%%%%%%%%%%%%%%%%%%%%%%%%%%%%%%%%%%%%%%%%%%%%%%%%%%%%%%%%%%%%%%%%%%%%%%%%

\subsection{$s$-coordinate transformation}
\label{section_annexe2}
The present appendix gathers several formula and relations essential to the development of the numerical implementation of the \textit{ocean model}.
%
%%%%%%%%%%%%%%%%%%%%%%%%%%%%%%%%%%%%%%%%%%%%%%%%%%%%%%%%%%%%%%%%%%%%%%%%%%%%%
\subsubsection{Transformation matrices}
\label{annexe_coordS}
%%%%%%%%%%%%%%%%%%%%%%%%%%%%%%%%%%%%%%%%%%%%%%%%%%%%%%%%%%%%%%%%%%%%%%%%%%%%%
The transformation matrix of the generalized coordinate transformation is:
\begin{equation}
    \displaystyle
    \Lambda^z_s=
    \begin{pmatrix}
    1 & 0 & 0 & 0 \\
    0 & 1 & 0 & 0 \\
    0 & 0 & 1 & 0 \\
    \frac{\partial z}{\partial t} & \frac{\partial z}{\partial x}
    & \frac{\partial z}{\partial y} & h=\frac{\partial z}{\partial s}
    \end{pmatrix}
\end{equation}
and the inverse transformation is given by:
\begin{equation}
    \displaystyle
    \Lambda_z^s=
    \begin{pmatrix}
    1 & 0 & 0 & 0 \\
    0 & 1 & 0 & 0 \\
    0 & 0 & 1 & 0 \\
    \frac{\partial s}{\partial t} & \frac{\partial s}{\partial x}
    & \frac{\partial s}{\partial y} & \frac{\partial s}{\partial z}
    \end{pmatrix}
\end{equation}
The Jacobian of the transformation $J=det(\Lambda^z_s)$ is the (specific) thickness:
\begin{equation}
 \displaystyle
 J=h=\frac{\partial z}{\partial s}=\frac{\partial z}{\partial s}\bigg\vert_{xyt}
\end{equation}
\cite{griffies_fundamentals_2004} further define the infinitesimal  thickness for modelling developments:
\begin{equation}
 \displaystyle
 \delta h=\frac{\partial z}{\partial s} \delta s
\end{equation}

%%%%%%%%%%%%%%%%%%%%%%%%%%%%%%%%%%%%%%%%%%%%%%%%%%%%%%%%%%%%%%%%%%%%%%%%%%%%%
\subsubsection{Formula and identities}
%%%%%%%%%%%%%%%%%%%%%%%%%%%%%%%%%%%%%%%%%%%%%%%%%%%%%%%%%%%%%%%%%%%%%%%%%%%%%
Base on the transformation matrices, the $s$-coordinate transformations can be rewritten:
\begin{subequations}
  \begin{alignat}{2}
  \displaystyle 
  &\frac{\partial A}{\partial t}\bigg\rvert_{xz} &&=
   \frac{\partial A}{\partial t}\bigg\rvert_{xs}
  - \frac{1}{h} \frac{\partial A}{\partial s}\bigg\rvert_{tx}
  \frac{\partial z}{\partial t}\bigg\rvert_{xs}\\[4mm]
  &\frac{\partial A}{\partial x}\bigg\rvert_{tz} &&=
   \frac{\partial A}{\partial x}\bigg\rvert_{ts}
  - \frac{1}{h} \frac{\partial A}{\partial s}\bigg\rvert_{tx}
  \frac{\partial z}{\partial x}\bigg\rvert_{ts}\\[4mm]
  &\frac{\partial A}{\partial z}\bigg\rvert_{tx} &&=
   \frac{1}{h}
   \frac{\partial A}{\partial s}\bigg\rvert_{tx}
  \end{alignat}
\end{subequations}
whereas material derivatives satisfy:
\begin{subequations}
  \begin{alignat}{2}
  \displaystyle 
  & \frac{d}{dt} &&=\frac{\partial}{\partial t}\bigg\vert_z
  + \mathbf{u}.\mathbf{\nabla}_z
  + w\frac{\partial }{\partial z}\\[4mm]
  & &&=\frac{\partial}{\partial t}\bigg\vert_s
  + \mathbf{u}.\mathbf{\nabla}_s
  + \dot{s}\frac{\partial}{\partial s}
  \end{alignat}
\end{subequations}
This leads to:
\begin{subequations}
  \begin{alignat}{2}
  \displaystyle 
  & \dot{z} &&=\frac{dz}{dt}\bigg\vert_s=\frac{\partial z}{\partial t}\bigg\vert_s
  + \mathbf{u}.\mathbf{\nabla}_s z
  + \dot{s}\frac{\partial z}{\partial s}\\[4mm]
  & \dot{s} &&=\frac{ds}{dt}\bigg\vert_z=\frac{\partial s}{\partial t}\bigg\vert_z
  + \mathbf{u}.\mathbf{\nabla}_z s
  + w\frac{\partial s}{\partial z}
  \end{alignat}
\end{subequations}
Using the identities:
\begin{subequations}
  \begin{alignat}{2}
  \displaystyle
  &\frac{\partial s}{\partial t}\bigg\vert_z &&=
  \left(\frac{\partial t}{\partial s}\bigg\vert_z\right)^{-1}\\[4mm]
  &\frac{\partial s}{\partial x}\bigg\vert_z &&=
  \left(\frac{\partial x}{\partial s}\bigg\vert_z\right)^{-1}\\[4mm]
  &\frac{\partial s}{\partial y}\bigg\vert_z &&=
  \left(\frac{\partial y}{\partial s}\bigg\vert_z\right)^{-1}\\[4mm]
  &\frac{\partial s}{\partial z}\bigg\vert_x &&=
  \left(\frac{\partial z}{\partial s}\bigg\vert_x\right)^{-1}
  \end{alignat}
\end{subequations}
several relations can be obtained from the triple product rule and the coordinate transformations are given by:
\begin{subequations}
  \begin{alignat}{2}
  \displaystyle
  &\frac{\partial z}{\partial t}\bigg\vert_s &&=
  -\frac{\partial s}{\partial t}\bigg\vert_z\frac{\partial z}{\partial s}\bigg\vert_s\\[4mm]
  &\frac{\partial z}{\partial x}\bigg\vert_s &&=
  -\frac{\partial s}{\partial x}\bigg\vert_z\frac{\partial z}{\partial s}\bigg\vert_s\\[4mm]
  &\frac{\partial z}{\partial y}\bigg\vert_s &&=
  -\frac{\partial s}{\partial y}\bigg\vert_z\frac{\partial z}{\partial s}\bigg\vert_s\\
  \end{alignat}
\end{subequations}

%%%%%%%%%%%%%%%%%%%%%%%%%%%%%%%%%%%%%%%%%%%%%%%%%%%%%%%%%%%%%%%%%%%%%%%%%%%%%
\subsubsection{Local orthonormal coordinates}
%%%%%%%%%%%%%%%%%%%%%%%%%%%%%%%%%%%%%%%%%%%%%%%%%%%%%%%%%%%%%%%%%%%%%%%%%%%%%
\cite{griffies_fundamentals_2004} further defines in his chapter (6.4) a system of orthonormal coordinates:
\begin{subequations}
  \begin{alignat}{2}
  \displaystyle 
  &\mathbf{e}_{x^*} &&=\frac{\mathbf{y}\wedge{\mathbf{\nabla}s}}
  {\norm{\mathbf{y}\wedge{\mathbf{\nabla}s}}}\\[4mm]
  &\mathbf{e}_{y^*} &&=\mathbf{e}_s\wedge{\mathbf{e}_{x^*}}\\[4mm]
  &\mathbf{e}_s &&=\frac{\mathbf{\nabla}s}{\norm{\mathbf{\nabla}s}}
  \end{alignat}
\end{subequations}
In this particular case ($\mathbf{e}_s.\mathbf{z}$) has a unique sign, the basis vectors can be rewritten:
\begin{subequations}
  \begin{alignat}{2}
  \displaystyle 
  &\mathbf{e}_{x^*} &&=\frac{\mathbf{x}+S_x\mathbf{z}}{\sqrt{1+S_x^2}}\\[4mm]
  &\mathbf{e}_{y^*} &&=\frac{-S_xS_y\mathbf{x}+(1+S_x^2)\mathbf{y}+S_y\mathbf{z}}{\sqrt{1+S^2)(1+S_x^2)}}\\[4mm]
  &\mathbf{e}_s &&=\frac{(-\mathbf{S},1)}{\sqrt{1+S^2}}
  \end{alignat}
\end{subequations}
The s-coordinate transformation is a rotation and:
\begin{equation}
   \displaystyle
   \mathbf{e}_{x^*y^*s}=\Lambda_{s}^{z}\mathbf{e}_{xyz}
\end{equation}
Note in particular the definition of the slope $\mathbf{S}$ and its norm $S=\norm{\mathbf{S}}$ used to rewrite the orthonormal basis:
\begin{equation}
   \displaystyle
   \mathbf{S}=\mathbf{\nabla}_s z=
   -\frac{\partial z}{\partial s}\mathbf{\nabla}_z s=\left( S_x,\ S_y,\ 0 \right)
\end{equation}
where $\mathbf{\nabla}_s z$ is "the horizontal gradient of the height of a fluid parcel as taken along surfaces of constant generalized vertical coordinate s" \citep{griffies_fundamentals_2004}.

Note that this orthonormal basis is not used to project the equations of the model. S-coordinates are "only" used as a change of variable whereas equations and vector quantities remain written in the original Cartesian or spherical basis. The present s-coordinate orthonormal basis is presented here to be latter used in the computation of fluxes through s-surfaces.


%%%%%%%%%%%%%%%%%%%%%%%%%%%%%%%%%%%%%%%%%%%%%%%%%%%%%%%%%%%%%%%%%%%%%%%%%%%%%
\subsection{Operators \& relations in s-coordinates}
\label{annexe_s-coord}
%%%%%%%%%%%%%%%%%%%%%%%%%%%%%%%%%%%%%%%%%%%%%%%%%%%%%%%%%%%%%%%%%%%%%%%%%%%%%

%%%%%%%%%%%%%%%%%%%%%%%%%%%%%%%%%%%%%%%%%%%%%%%%%%%%%%%%%%%%%%%%%%%%%%%%%%%%%
\subsubsection{Divergence of the velocity field in s-coordinates}
%%%%%%%%%%%%%%%%%%%%%%%%%%%%%%%%%%%%%%%%%%%%%%%%%%%%%%%%%%%%%%%%%%%%%%%%%%%%%
Using :
\begin{equation}
 \displaystyle
 \frac{\partial}{\partial t} \frac{\partial z}{\partial s}\bigg\vert_{tx}= \frac{\partial h}{\partial t} \qquad and \qquad \frac{\partial}{\partial x} \frac{\partial z}{\partial s}\bigg\vert_{tx}= \frac{\partial h}{\partial x}
\end{equation}
%and
%\begin{equation}
% \displaystyle
% \frac{\partial}{\partial x} \frac{\partial z}{\partial s}\bigg\vert_{tx}= \frac{\partial h}{\partial x}
%\end{equation}
%
the expression of the divergence of the velocity field in s-coordinates can be written:
\begin{subequations}
  \begin{alignat}{2}
  & h \ \mathbf{\nabla}.( \mathbf v) &&= h \frac{\partial u}{\partial x} \bigg \rvert_{zt} +h \frac{\partial v_z}{\partial z} \bigg \rvert_{xt}\\[4mm]
  & && = h \frac{\partial u}{\partial x} \bigg \rvert_{st} - \frac{h}{h} \frac{\partial u}{\partial s}\bigg \rvert_{tx} \frac{\partial z}{\partial x}\bigg \rvert_{ts} \\[4mm]
  & && \quad + \frac{h}{h}  \frac{\partial}{\partial s} \bigg ( v_s + \frac{\partial z }{\partial t}\bigg \rvert_{xs} + u \frac{\partial z}{\partial x}\bigg \rvert_{ts} \bigg )\\[4mm]
  & && = h \frac{\partial u}{\partial x} \bigg \rvert_{st} -  \frac{\partial u}{\partial s}\bigg \rvert_{tx} \frac{\partial z}{\partial x}\bigg \rvert_{ts} \\[4mm]
  & && \quad +  \frac{\partial v_s}{\partial s} +  \frac{\partial h}{\partial t} + u \frac{\partial h}{\partial x} + \frac{\partial u}{\partial s}\bigg \rvert_{tx} \frac{\partial z}{\partial x}\bigg \rvert_{ts}\\[4mm]
  & && = \frac{\partial v_s}{\partial s}\bigg \rvert_{tx} + \frac{\partial h u}{\partial x} \bigg \rvert_{ts}+ \frac{\partial h}{\partial t}\bigg \rvert_{xs}
  \end{alignat}
\end{subequations}
Note that this is a particular case of the formulation of a change of variables with its Jacobian ($J=h$ in the present case). %This leads to several useful conservative formulations in the following section.
%
%%%%%%%%%%%%%%%%%%%%%%%%%%%%%%%%%%%%%%%%%%%%%%%%%%%%%%%%%%%%%%%%%%%%%%%%%%%%%
\subsubsection{Conservative "flux" forms: kinematics \& dynamics}
%%%%%%%%%%%%%%%%%%%%%%%%%%%%%%%%%%%%%%%%%%%%%%%%%%%%%%%%%%%%%%%%%%%%%%%%%%%%%
Two general conservative formulations can be obtained combining this with the continuity equation \citep{auclair_woceanfr_2011}\footnote{WOcean.fr Web Site: \url{http://poc.omp.obs-mip.fr/auclair/WOcean.fr/SNH/Restricted/NH-NBQ/Sources/Images/png/Coord_demo.png}\label{WOcean_scoord}}.

$A$ is a property given per unit mass (thermodynamically intensive) (see the demonstration on web site). The first two (conservative) relations are fundamentals to analytical and numerical modeling.


\textbf{\textit{Based on the conservation of mass:}}
\begin{equation}
  \displaystyle 
  \rho \frac{d A}{dt}
  =\frac{\partial \rho A}{\partial t}\bigg\rvert_{xz}
  +\frac{\partial \rho A u}{\partial x}\bigg\rvert_{tz}
  +\frac{\partial \rho  v_s}{\partial z}\bigg\rvert_{tx}
\end{equation}

\textbf{\textit{Based on the conservation of mass \& in s-coordinates:}}
\begin{equation}
  \displaystyle 
  \rho h \frac{d A}{dt}
  =\frac{\partial \rho h A}{\partial t}\bigg\rvert_{xs}
  +\frac{\partial \rho h A u}{\partial x}\bigg\rvert_{ts}
  +\frac{\partial \rho  A v_s}{\partial s}\bigg\rvert_{tx}
\end{equation}
\textbf{\textit{A kinematic, non-conservative formulation}} can be obtained without the continuity equation:
\begin{equation}
\frac{d A}{d t} = \frac{\partial A}{\partial t} \bigg\rvert_{xs} + u \frac{\partial A}{\partial x} \bigg\rvert_{ts} + \frac{v_s}{h}\frac{\partial A}{\partial s}\bigg\rvert_{tx}
\end{equation}
The demonstration is given in \citep{auclair_woceanfr_2011}$^{\noparref{WOcean_scoord}}$.\\

\textbf{\textit{Conservation of mass:}}
note finally that the conservation of mass $A=1$ can then be rewritten:
\begin{equation}
  \displaystyle 
  \label{mass_s}
  h\frac{d\rho}{d t}
  =\frac{\partial \rho h }{\partial t}\bigg\rvert_{xs}
  +\frac{\partial \rho h u}{\partial x}\bigg\rvert_{ts}
  +\frac{\partial \rho  v_s}{\partial s}\bigg\rvert_{tx}
\end{equation}

Additionnally, the evolution of $\rho$ in equation \ref{eq_diff_cart} can be rewritten in s-coordinates as:
\begin{equation}
\label{eq_diff_s}
\displaystyle
h \frac{d \rho }{d t} \approx
\frac{\partial}{\partial x} \bigg(h \kappa^h \frac{\partial \rho}{\partial x}\bigg\rvert_{ts}\bigg)_{ts}
+ \frac{\partial}{\partial s} \bigg(\frac{\kappa^v}{h} \frac{\partial \rho}{\partial s}\bigg\rvert_{tx}\bigg)_{tx} 
\end{equation}
with: $\kappa_c^h \approx \kappa^h$ and $\kappa_c^v \approx \kappa^v$.\\



%%%%%%%%%%%%%%%%%%%%%%%%%%%%%%%%%%%%%%%%%%%%%%%%%%%%%%%%%%%%%%%%%%%%%%%%%%%%%
\subsection{Reynolds Theorem}
\label{annexe_reynolds}
%%%%%%%%%%%%%%%%%%%%%%%%%%%%%%%%%%%%%%%%%%%%%%%%%%%%%%%%%%%%%%%%%%%%%%%%%%%%%
\subsubsection{Flux through a surface}
%%%%%%%%%%%%%%%%%%%%%%%%%%%%%%%%%%%%%%%%%%%%%%%%%%%%%%%%%%%%%%%%%%%%%%%%%%%%%
Based on \citep{delhaye_thermohydraulique_2008}, let $\mathbf{v}_{\Sigma}$ be the velocity of a material surface in the fluid:
\begin{equation}
	\displaystyle
	\mathbf{v}_{\Sigma}=\frac{\partial \mathbf{x}}{\partial t}\bigg\rvert _{\Sigma}
\end{equation}
with $\mathbf{x}$ the position of the current point where the velocity is computed. This velocity is undetermined  since the tangential component is itself undetermined. Only its normal component $(\mathbf{n}_{\Sigma}.\mathbf{v}_{\Sigma})$ can be computed and isolated. This component is different from the vertical component of $\mathbf{v}_{\Sigma}$ of equation \ref{eq_vertvelcomp}.

Demonstrations and further expressions of the normal component can be found in \citet{griffies_fundamentals_2004}, \citet{griffies_elements_2012} and \citet{delhaye_thermohydraulique_2008}.
\begin{subequations}
  \begin{alignat}{2}
  \displaystyle 
  & \frac{\partial z}{\partial t}\bigg\vert_s &&=
  \frac{\partial s}{\partial t}\bigg\vert_z
  \frac{\partial z}{\partial s}\bigg\vert_t\\
  & &&=h\norm{\mathbf{\nabla}s}\mathbf{n}_{\Sigma}.
  \mathbf{v}_{\Sigma}
  \end{alignat}
\end{subequations}
Metric tensor formulations lead to:
\begin{subequations}
  \begin{alignat}{2}
 & dS_{\Sigma} &&= h\norm{\mathbf{\nabla}s} dS\\
 & h dS &&=\mathbf{n}_{\Sigma}.\mathbf{v}_{\Sigma}dS_{\Sigma}
  \end{alignat}
\end{subequations}
%See bellow how this formula is used to write Leibniz rule...

\subsubsection{Formulations of the Reynolds Transport Theorem (\& Leibniz Rule)}
\paragraph{Demonstration.}
The difficulty to compute:
\begin{equation}
 \displaystyle
 \frac{d}{dt} \iiint_{\mathcal{V}_M(t)} \rho A d\tau
\end{equation}
is due to the fact that for a general time-varying volume $\mathcal{V}(t)$ the differentiation cannot be taken through the internal sign. To do so, we can however consider a change of coordinates to volume-attached coordinates \citep{hirasaki_chapter_2021}. Let $\mathcal{J}$ be the Jacobian (determinant of the Jacobian matrix) of the change of variables from Cartesian coordinates $(\mathbf{x},t)$ to volume varying coordinates $(\boldsymbol{\xi},t)$. Integrating volume is $\mathcal{V}(t)=\mathcal{V}_{\xi}=\mathcal{V}_{\xi}(t)$.

Kinematic evolution of material particulate ("dilation kinematic theorem") is then nothing but Reynolds'transport theorem (which is equivalent to Leibniz derivation rule to 3D volumes) \citep{hirasaki_chapter_2021}. Caution: the demonstration given in \citep{hirasaki_chapter_2021} is for a material volume $\mathcal{V}_M$. As the demonstration is kinematic, it is generalized here to a general volume moving thought the fluid flow $(\mathcal{V}(t))$. This problem is dealt with in \cite{web_course_web_2021} (and further references are given in there).

As a consequence for a general volume $\mathcal{V}(t)$ moving at a 3D velocity $  \mathbf{v}_{\Sigma}$:
\begin{equation}
 \displaystyle
 \frac{dJ}{dt}=J \mathbf{\nabla}.\mathbf{v}_{\Sigma}
\end{equation}
which is reminiscent of the kinematic evolution of the specific volume of a fluid particle:
\begin{equation}
 \displaystyle
 \frac{d\tau}{dt}=\tau \mathbf{\nabla}.\mathbf{v}
\end{equation}
sometime written:
\begin{equation}
 \displaystyle
 \frac{d\delta\tau}{dt}=\delta\tau \mathbf{\nabla}.\mathbf{v}
\end{equation}
which is noting but the continuity equation written for the specific volume. With the natural change of variable, the derivation can enter the integral sign and:
\begin{subequations}
  \begin{alignat}{2}
  \displaystyle 
  & \frac{d}{dt} \iiint_{\mathcal{V}(t)} \rho A d\tau &&=
  \frac{D}{Dt} \iiint_{\mathcal{V}(t)} \rho A d\tau\\[4mm]
  & &&= \frac{D}{Dt} \iiint_{\mathcal{V}(t)} \rho A d\mathbf{x}\\[4mm]
  & &&=\frac{D}{Dt} \iiint_{\mathcal{V}_{\xi}} \rho A J d\boldsymbol{\xi}\\[4mm]
  & &&=\iiint_{\mathcal{V}_{\xi}} \frac{D\rho A}{Dt}  J d\boldsymbol{\xi}
  +\iiint_{\mathcal{V}_{\xi}} \rho A \frac{DJ}{Dt} d\boldsymbol{\xi}\\[4mm]
  & &&=\iiint_{\mathcal{V}_{\xi}} \frac{D\rho A}{Dt}  J d\boldsymbol{\xi}
  +\iiint_{\mathcal{V}_{\xi}} \rho A \mathbf{\nabla}. \mathbf{v}_{\Sigma}\ J d\boldsymbol{\xi}
  \end{alignat}
\end{subequations}
Following \cite{truesdell_classical_1960}, a more general formulation can thus be recovered for a time-dependant volume $\mathcal{V}(t)$ moving with a velocity $\mathbf{v}_{\Sigma}$:
\begin{subequations}
  \begin{alignat}{2}
  \displaystyle 
  &  \frac{d}{dt} \iiint_{\mathcal{V}(t)} \rho A d\tau && =
  \iiint_{\mathcal{V}(t)} \frac{D\rho A}{Dt}  d\boldsymbol{x}
  +\iiint_{\mathcal{V}(t)} \rho A \mathbf{\nabla}.\mathbf{v}_{\Sigma}\ d\mathbf{x}\\[4mm]
 & && =
  \iiint_{\mathcal{V}(t)} \frac{\partial \rho A}{\partial t}\bigg\rvert_{xz} d\tau
  %+ \varoiint_{\mathcal{V}(t)}\rho A \mathbf{v}.\mathbf{S}\\
  + \oiint_{\mathcal{S}(t)}\rho A   \mathbf{v}_{\Sigma}.\mathbf{n}_{\Sigma}dS_{Sigma}
  \end{alignat}
\end{subequations}
where:
\begin{equation}
 \displaystyle
 \frac{D\bullet}{Dt}=\frac{\partial \bullet}{\partial t}\bigg\vert_{\boldsymbol{\xi}}
 +  \mathbf{v}_{\Sigma}.\mathbf{\nabla}_{t,\boldsymbol{\xi}}\bullet
\end{equation}
This is the generalized (Reynolds) transport theorem or the generalized Leibnitz theorem. Caution: the derivative $D/Dt$ is associated to $  \mathbf{v}_{\Sigma}$.

\paragraph{Closed time-dependent volume $\mathcal{V}(t)$.} 
Several formulations of this kinematic theorem can be given:
\begin{subequations}
  \begin{alignat}{2}
  \displaystyle 
  & \frac{d}{dt} \iiint_{\mathcal{V}(t)} \rho A d\tau &&=
  \iiint_{\mathcal{V}(t)} \frac{\partial \rho A}{\partial t}\bigg\rvert_{xz} d\tau
  %+ \varoiint_{\mathcal{V}(t)}\rho A \mathbf{v}.\mathbf{S}\\
  + \oiint_{\mathcal{V}(t)}\rho A   \mathbf{v}_{\Sigma}.d\mathbf{S}\\[4mm]
  % Line 2:
  &  &&= \iiint_{\mathcal{V}(t)} \left(\frac{\partial \rho A}{\partial t}\bigg\rvert_{xz}
  +\mathbf{\nabla}.(\rho A   \mathbf{v}_{\Sigma}) \right) d\tau\\[4mm]
  % Line 3:
  &  &&= \iiint_{\mathcal{V}(t)} \left(\rho \frac{\partial A}{\partial t}\bigg\rvert_{xz}
  +A \frac{ \partial \rho }{\partial t}\bigg\rvert_{xz}
  +\rho   \mathbf{v}_{\Sigma}.\mathbf{\nabla}A 
  +A\mathbf{\nabla}.(\rho   \mathbf{v}_{\Sigma}) \right) d\tau\\[4mm]
  % Line 4:
  & &&=\iiint_{\mathcal{V}(t)} \rho \frac{DA}{Dt}  d\tau 
  +\iiint_{\mathcal{V}(t)} A \left( \frac{\partial \rho}{\partial t} \bigg\rvert_{xz}
  +\mathbf{\nabla}.(\rho  \mathbf{v}_{\Sigma})\right) \ d\tau
  \end{alignat}
\end{subequations}
This can also be written:
\begin{subequations}
  \begin{alignat}{2}
  \displaystyle 
  & \frac{d}{dt} \iiint_{\mathcal{V}(t)} \rho A d\tau &&=
   \iiint_{\mathcal{V}(t)} \left(\rho \frac{\partial A}{\partial t}\bigg\rvert_{xz}
  +A \frac{ \partial \rho }{\partial t}\bigg\rvert_{xz}
  +\rho   \mathbf{v}_{\Sigma}.\mathbf{\nabla}A 
  +A   \mathbf{v}_{\Sigma}.\mathbf{\nabla}\rho
  +A\rho\mathbf{\nabla}.  \mathbf{v}_{\Sigma} \right) d\tau\\[4mm]
  % Line 2:
  & &&=\iiint_{\mathcal{V}(t)} \rho \frac{DA}{Dt}  d\tau 
  +\iiint_{\mathcal{V}(t)} A \frac{D \rho}{D t} d\tau
  + \iiint_{\mathcal{V}(t)} \rho A\mathbf{\nabla}.  \mathbf{v}_{\Sigma} d\tau
  \end{alignat}
\end{subequations}
or alternatively:
\begin{subequations}
  \begin{alignat}{2}
  \displaystyle 
  & \frac{d}{dt} \iiint_{\mathcal{V}(t)} \rho A d\tau &&
  =\iiint_{\mathcal{V}(t)} \rho \frac{dA}{dt}  d\tau 
  +\iiint_{\mathcal{V}(t)} \left(A \frac{d \rho}{dt} 
  +\mathbf{\nabla}.\left[ \rho A(  \mathbf{v}_{\Sigma}-\mathbf{v})\right]
  +\rho A \mathbf{\nabla}.\mathbf{v} \right) d\tau\\[4mm]
  & &&
  =\iiint_{\mathcal{V}(t)} \rho \frac{dA}{dt}  d\tau 
  + \iiint_{\mathcal{V}(t)}\mathbf{\nabla}.\left[ \rho A(  \mathbf{v}_{\Sigma}-\mathbf{v})\right]d\tau
    \end{alignat}
\end{subequations}
where $  \mathbf{v}_{\Sigma}$ is defined as the velocity of the closed surface of $\mathcal{V}(t)$ and is more generally the 3D displacement velocity of the volume $\mathcal{V}(t)$.

 Caution: this has nothing to do with the velocity of the fluid in this volume except in the following particular case when considering a material volume, i.e. in the Lagrangian approach.
These general formulations and demonstrations are useful in s-coordinates.

\subsubsection{Material volume $\mathcal{V}_M(t)$ (kinematics \& dynamics)}
\paragraph{Formulation of the transport theorem.} 
A material volume $\mathcal{V}_M(t)$ is defined as a volume containing exactly the same fluid particles as time goes on.
Simplifications to obtain the simple expression of the integral variations in the case of a moving material volume are based on the use of the continuity equation or conservation of mass. 

As the control volume is advected by mean flow (Lagrangian approach), $  \mathbf{v}_{\Sigma}=\mathbf{v}$ :
\begin{equation}
  \displaystyle 
   \frac{d}{dt} \iiint_{\mathcal{V}_M(t)} \rho A d\tau =
  \iiint_{\mathcal{V}_M(t)} \rho \frac{dA}{dt}  d\tau 
  +\iiint_{\mathcal{V}_M(t)} A \frac{d \rho}{dt} d\tau
  +\iiint_{\mathcal{V}_M(t)} \rho A\mathbf{\nabla}.\mathbf{v} d\tau
\end{equation}
and thus:
\begin{subequations}
  \begin{alignat}{2}
  \displaystyle 
  & \frac{d}{dt} \iiint_{\mathcal{V}_M(t)} \rho A d\tau &&= \iiint_{\mathcal{V_M}(t)} \rho \frac{dA}{dt}  d\tau 
  +\iiint_{\mathcal{V}_M(t)} A \underbrace{\left( \frac{\partial \rho}{\partial t} \bigg\rvert_{xz}+\mathbf{\nabla}.(\rho\mathbf{v})\right)}_{=0} \ d\tau\\[4mm]
  & && = \iiint_{\mathcal{V}_M(t)} \rho \frac{dA}{dt}  d\tau 
  \end{alignat}
\end{subequations}
This is a generalization to 3D volume of the Leibniz rule. The time dependency of the integral volume $\mathcal{V}_M(t)$ is for a Lagrangian evolution with time-dependent boundaries. The present formulation is to be used for time-dependent boundaries whatever the velocity $  \mathbf{v}_{\Sigma}$ inside the volume.

\paragraph{Material volume $\mathcal{V}_M(t)$ in Lagrangian material coordinates.} 
The same exact approach and demonstration can be carried out in the case of a material volume as in the general case.
Let in this case $\mathcal{J}$ be the Jacobian (determinant of the Jacobian matrix) of the change of variables from Cartesian coordinates $(\mathbf{x},t)$ to material coordinates $(\boldsymbol{\xi},t)$. Integrating volume is $\mathcal{V}(t)=\mathcal{V}_{\xi}$.\\
Kinematic evolution of material particle ("dilation kinematic theorem") reads \citep{hirasaki_chapter_2021}:
\begin{equation}
 \displaystyle
 \frac{dJ}{dt}=J \mathbf{\nabla}.\mathbf{v}
\end{equation}
which is reminiscent of the kinematic evolution of a control volume:
\begin{equation}
 \displaystyle
 \frac{d\tau}{dt}=\tau \mathbf{\nabla}.\mathbf{v}
\end{equation}
With this kinematic theorem:
\begin{subequations}
  \begin{alignat}{2}
  \displaystyle 
  & \frac{d}{dt} \iiint_{\mathcal{V}_M(t)} \rho A d\tau &&=
  \frac{d}{dt} \iiint_{\mathcal{V}_M(t)} \rho A d\mathbf{x}\\[4mm]
  & &&=\frac{d}{dt} \iiint_{\mathcal{V}_{s}} \rho A J d\boldsymbol{\xi}\\[4mm]
  & &&=\iiint_{\mathcal{V}_{\xi}} \frac{d\rho A}{dt}  J d\boldsymbol{\xi}
  +\iiint_{\mathcal{V}_{\xi}} \rho A \frac{dJ}{dt} d\boldsymbol{\xi}\\[4mm]
  & &&=\iiint_{\mathcal{V}_{\xi}} \frac{d\rho A}{dt}  J d\boldsymbol{\xi}
  +\iiint_{\mathcal{V}_{\xi}} \rho A \mathbf{\nabla}.\mathbf{v}\ J d\boldsymbol{\xi}
  \end{alignat}
\end{subequations}
This leads finally to:
\begin{subequations}
  \begin{alignat}{2}
  \label{dintdt_s}
  \displaystyle 
  & \frac{d}{dt} \iiint_{\mathcal{V}_M(t)} \rho A d\tau &&=
  \iiint_{\mathcal{V}_{\xi}}\left( \frac{\partial \rho A}{\partial t}   
  +\mathbf{\nabla}.( \rho A \mathbf{v})\right) J d\boldsymbol{\xi} \\[4mm]
  & && =\iiint_{\mathcal{V}_M(t)} \frac{\partial \rho A}{\partial t}\bigg\rvert_{xz} d\tau
  + \oiint_{\mathcal{V}_M(t)}\rho A \mathbf{v}.d\mathbf{S}
  \end{alignat}
\end{subequations}
Reynolds transport theorem is recovered for a material volume, understood as a volume containing  the same fluid particles. This last relation can be viewed as the difference between the time variation of the extensive variable inside the domaine and the advective flux through the "control" surface matching at time "t" with material volume.

\paragraph{Time-independant "control" volume $\mathcal{V}_0$.} 
In the particular case when the volume (and thus his surface) is independent of time, then $  \mathbf{v}_{\Sigma}=0$ and as a consequence:
\begin{equation}
 \displaystyle
 	\frac{d}{dt}\iiint_{\mathcal{V}_0} A d\tau = \iiint_{\mathcal{V}_0}\frac{\partial A}{\partial t} d\tau
\end{equation}

\subsection{Formulations based on total derivatives (toward Lagrangian relations...)}
Note that the continuity equation can be written:
\begin{equation}
	\displaystyle
	\frac{d\ \delta \tau}{dt}=\frac{\partial v_{k}}{\partial x_k} \delta\tau= \mathbf{\nabla}.\mathbf{v}\ \delta\tau
\end{equation}

and so for a closed, moving volume $\mathcal{V}(t)$:
\begin{subequations}
  \begin{alignat}{2}
  \displaystyle 
  & \frac{d}{dt} \iiint_{\mathcal{V}(t)} \rho A d\tau &&=  
  \iiint_{\mathcal{V}(t)} \rho \frac{DA}{Dt}  d\tau 
  +\iiint_{\mathcal{V}(t)} A \frac{D(\rho d\tau)}{Dt}\\[4mm]
  & && =  \iiint_{\mathcal{V}(t)} \rho \frac{DA}{Dt}  d\tau 
  + \iiint_{\mathcal{V}(t)} A \frac{D\rho}{Dt} d\tau
  +\iiint_{\mathcal{V}(t)} \rho A \frac{Dd\tau}{Dt}\\[4mm]
  & && = \iiint_{\mathcal{V}(t)} \rho \frac{DA}{Dt}  d\tau 
  +\iiint_{\mathcal{V}(t)} A \frac{D\rho}{Dt} d\tau
  +\iiint_{\mathcal{V}(t)} \rho A \mathbf{\nabla}.  \mathbf{v}_{\Sigma}\ d\tau \\[4mm]
  & && = \iiint_{\mathcal{V}(t)} \rho \frac{DA}{Dt}  d\tau 
  \end{alignat}
\end{subequations}
where $D\bullet/Dt=\partial \bullet/\partial t+  \mathbf{v}_{\Sigma}.\mathbf{\nabla}\bullet$. The last simplification is associated to the conservation of mass when written:
\begin{equation}
 \displaystyle
 \frac{D\rho}{Dt}=-\rho\mathbf{\nabla}.  \mathbf{v}_{\Sigma}
\end{equation}
If the volume is now a material volume and is thus advected by the mean flow (Lagrangian approach), $  \mathbf{v}_{\Sigma}=\mathbf{v}_M$ and $D./Dt =d./dt$ :
\begin{subequations}
  \begin{alignat}{2}
  \displaystyle 
  &  \frac{d}{dt} \iiint_{\mathcal{V}_M(t)} \rho A d\tau && = 
  \iiint_{\mathcal{V}_M(t)} \rho \frac{dA}{dt}  d\tau 
  +\iiint_{\mathcal{V}_M(t)} A \frac{d\rho}{dt} d\tau
  +\iiint_{\mathcal{V}_M(t)} \rho A \mathbf{\nabla}.\mathbf{v} d\tau\\[4mm]
  & && \left[ =\iiint_{\mathcal{V}_M(t)} \rho \frac{dA}{dt}  d\tau 
  +\iiint_{\mathcal{V}_M(t)} A  \underbrace{\left(\frac{\partial \rho}{\partial t} 
  +\underbrace{\mathbf{v}.\mathbf{\nabla}\rho+\rho \mathbf{\nabla}.\mathbf{v} }_{=\mathbf{\nabla}.(\rho\mathbf{v})}\right)}_{=0} d\tau \right]\\[4mm]
  & && = \iiint_{\mathcal{V}_M(t)} \rho \frac{dA}{dt}  d\tau 
  +\iiint_{\mathcal{V}_M(t)} A \underbrace{\left( -\rho \mathbf{\nabla}.\mathbf{v} + \rho \mathbf{\nabla}.\mathbf{v}\right)}_{=0} d\tau\\[4mm]
  & && = \iiint_{\mathcal{V}_M(t)} \rho \frac{dA}{dt}  d\tau 
  \end{alignat}
\end{subequations}
recovering  the results from the Eulerian approach and the previous more general formulation.

\subsubsection{Rate of change of integrals in s-coordinate}

In s-coordinates, the ocean column behaves as a material volume in the vertical direction but not in the horizontal direction where it a fixed-in-time control volume. For a volume associated to the s-coordinate grid: $  \mathbf{v}_{\Sigma}=\mathbf{v}_h+\mathbf{v}_z-\mathbf{v}_s$. This is the vertical velocity of the iso-s surfaces. The integration volume in s-coordinates is a sum of "waters columns" and is named $\mathcal{V}_s$. 

\paragraph{Kinematic-Dynamic evolution (Left-Hand-Side).}
In an Eulerian approach, the depth-to-surface integration can be carried out using \ref{mass_s} with $J=h(t,\mathbf{x},z)$:
\begin{subequations}
  \begin{alignat}{2}
  \displaystyle 
 	&\frac{d }{d t} \iiint_{\mathcal{V}_{s}} \rho A\ d\tau &&
   =\iiint_{\mathcal{V}_{s}} \frac{d \rho A }{d t}\ d\tau \\[4mm]
   & && = \iiint_{\mathcal{V}_{s}} \frac{\partial \rho A}{\partial t}d\tau
  + \iiint_{\mathcal{V}_{s}}\mathbf{\nabla}.( \rho A   \mathbf{v}_{\Sigma}) d\tau \\[4mm]
  & &&
   =\iiint_{\mathcal{V}_{s}} \frac{\partial \rho A}{\partial t}d\tau
 +\iint_{\mathcal{S}} \rho A\   \mathbf{v}_{\Sigma}.\mathbf{n}_{\Sigma}\ dS_{\Sigma}
  \end{alignat}
\end{subequations}
\cite{griffies_fundamentals_2004}, \cite{griffies_elements_2012} and \cite{delhaye_thermohydraulique_2008} provides a reformulation of the latest term in Cartesian coordinates:
\begin{equation}
%  \begin{alignat}{2}
  \displaystyle 
 	\frac{d }{d t} \iiint_{\mathcal{V}_{s}} \rho A\ d\tau
   =\iiint_{\mathcal{V}_{s}} \frac{\partial \rho A}{\partial t}\underbrace{d\tau}_{dxdydz}
 +\iint_{\mathcal{S}} \rho A\  \frac{\partial z}{\partial t}\bigg\vert_{s}\ \underbrace{dS}_{dxdy}
%  \end{alignat}
\end{equation}
using the central relation (2.121) of \citep{griffies_elements_2012} and in \S 6 of \citep{griffies_fundamentals_2004}:
\begin{equation}
 \displaystyle
  \mathbf{v}_{\Sigma}.\mathbf{n}_{\Sigma}\ dS_{\Sigma}= \frac{\partial z}{\partial t}\bigg\vert_{s}\ dS
\end{equation}
Since the ocean bottom boundary is supposed to remain stationary, this leads eventually to:
\begin{equation}
\label{eq_reyn_zeta}
%  \begin{alignat}{2}
  \displaystyle 
 	\frac{d }{d t} \iiint_{\mathcal{V}_{s}} \rho A\ d\tau 
   =\iiint_{\mathcal{V}_{s}} \frac{\partial \rho A}{\partial t}\underbrace{d\tau}_{dxdydz}
 +\iint_{\mathcal{S}_{surf}} \rho A\  \frac{\partial \zeta}{\partial t}\ \underbrace{dS}_{dxdy}
%  \end{alignat}
\end{equation}

\paragraph{Using the kinematic condition.}
Caution: to recover this relation based on the kinematic condition, one has to remember that $\mathbf{v}_{\Sigma}$ is indeterminate \citep{delhaye_thermohydraulique_2008}. Its component tangent to the iso-s surface depends in particular on the parametrization of this surface. The component orthogonal to this surface can however be computed. As a consequence, one has to consider the product $\mathbf{v}_{\Sigma}.\mathbf{n}_{\Sigma}$ rather than $\mathbf{v}_{\Sigma}$ alone. 

The previous relation can then be recovered writing:
\begin{subequations}
  \begin{alignat}{2}
  \displaystyle 
  % Line 2:
 &\frac{d }{d t} \iiint_{\mathcal{V}_{s}} \rho A\ d\tau &&=
 \iiint_{\mathcal{V}_{s}} \frac{\partial \rho A}{\partial t}d\tau
 +\iint_{\mathcal{S}} \rho A\   \mathbf{v}_{\Sigma}.\mathbf{n}_{\Sigma}\ dS_{\Sigma}\\[4mm]
&  &&=
 \iiint_{\mathcal{V}_{s}} \frac{\partial \rho A}{\partial t}d\tau
 +\iint_{\mathcal{S}_{surf}} \rho A\  ( \mathbf{u}_{\Sigma}-\mathbf{w}_{\Sigma}).\mathbf{n}_{\Sigma}\ dS_{\Sigma}\\[4mm]
&  &&=
 \iiint_{\mathcal{V}_{s}} \frac{\partial \rho A}{\partial t}d\tau
 +\iint_{\mathcal{S}_{surf}}  \frac{\rho A}{\norm{\mathbf{\nabla}s}}
 \left( \mathbf{\nabla}_z s.\mathbf{u}-\frac{h}{h}(\frac{\partial s}{\partial t}
 +\mathbf{\nabla}_z s.\mathbf{u})
 \right)\underbrace{\ dS_{\Sigma}}_{=\sqrt{1+S^2} dS}
 \end{alignat}
\end{subequations}
Since $h \norm{\mathbf{\nabla}s}=\sqrt{1+S^2}$, we can recover:
\begin{equation}
 \displaystyle
 \frac{d }{d t} \iiint_{\mathcal{V}_{s}} \rho A\ d\tau=
  \iiint_{\mathcal{V}_{s}} \frac{\partial \rho A}{\partial t}d\tau  
  +\iint_{\mathcal{S}_{surf}}  \rho A \frac{\partial s}{\partial t} dS
\end{equation}
Then knowing that:
\begin{subequations}
  \begin{alignat}{2}
 \displaystyle
&\frac{\partial s}{\partial t}&&=
\frac{\partial z}{\partial t}-
\frac{\partial \zeta}{\partial t}\\[4mm]
& &&=\frac{\partial \zeta}{\partial t}=\frac{\partial \zeta}{\partial t}\bigg\vert_s
 \end{alignat}
 we can conclude that:
\end{subequations}
\begin{equation}
 \displaystyle
 \frac{d }{d t} \iiint_{\mathcal{V}_{s}} \rho A\ d\tau=
  \iiint_{\mathcal{V}_{s}} \frac{\partial \rho A}{\partial t}d\tau  
  +\iint_{\mathcal{S}_{surf}}  \rho A \frac{\partial \zeta}{\partial t} dS
\end{equation}


\paragraph{In s-coordinates.} 
This result can also be recovered considering Reynolds transport theorem and the fact that $\mathbf{v}_{s}$ vanishes over both the bottom and surface boundaries\color{black}.
This latest formulation can be recovered from the expression in Cartesian coordinates since
\begin{equation}
  \displaystyle 
 	\frac{d }{d t} \iiint_{\mathcal{V}_{s}} \rho A\ d\tau  =
 	\iiint_{\mathcal{V}_{s}} \frac{\partial \rho A}{\partial t}d\tau
 +\iint_{\mathcal{S}} \rho A\  \frac{\partial z}{\partial t}\bigg\vert_s \ dS
\end{equation}
The first term on the right hand side can be rewritten:
\begin{subequations}
  \begin{alignat}{2}
  \displaystyle
 & \iiint_{\mathcal{V}_{s}} \frac{\partial \rho A}{\partial t}d\tau &&=
 \iiint_{\mathcal{V}_{s}} \left( \frac{\partial \rho A}{\partial t}\bigg \vert_s 
 -\frac{1}{h}\frac{\partial \rho A}{\partial s}\frac{\partial z}{\partial t}\bigg\vert_s \right) h \ ds dx dy\\[4mm]
 & &&=
 \iiint_{\mathcal{V}_{s}} \left( \frac{\partial \rho h A}{\partial t}\bigg \vert_s 
 -\rho A \frac{\partial h}{\partial t}\bigg\vert_s
 -\frac{\partial}{\partial s}\left(\rho A \frac{\partial z}{\partial t}\right) \bigg\vert_s
 +\rho A \underbrace{\frac{\partial^2 z}{\partial s\partial t}}_{\partial h/\partial t\vert_s}
 \right) ds dx dy
  \end{alignat}
\end{subequations}
Canceling the second, third and fourth terms together and using the Green-Ostrogradsky theorem lead to:
%\begin{subequations}
%  \begin{alignat}{2}
%  \displaystyle 
% 	&\frac{d }{d t} \iiint_{\mathcal{V}_{s}} \rho A\ d\tau  &&=
% \iiint_{\mathcal{V}_{s}} \left( \frac{\partial \rho h A}{\partial t}\bigg \vert_s 
% \pm\frac{1}{h}\frac{\partial}{\partial s}\left(\rho A \frac{\partial z}{\partial t}\right) \bigg\vert_s
% \right) h ds dx dy\\
% & &&= \iiint_{\mathcal{V}_{s}} \frac{\partial \rho h A}{\partial t}\bigg \vert_s h ds dx dy
%  \end{alignat}
%\end{subequations}
\begin{subequations}
  \begin{alignat}{2}
  \displaystyle 
 	&\frac{d }{d t} \iiint_{\mathcal{V}_{s}} \rho A\ d\tau  &&=
 \iiint_{\mathcal{V}_{s}} \left( \frac{\partial \rho h A}{\partial t}\bigg \vert_s 
 \pm \frac{\partial}{\partial s}\left(\rho A \frac{\partial z}{\partial t}\right) \bigg\vert_s
 \right) ds dx dy\\[4mm]
 & &&= \iiint_{\mathcal{V}_{s}} \frac{\partial \rho h A}{\partial t}\bigg \vert_s ds dx dy
  \end{alignat}
\end{subequations}

\paragraph{Lagrangian approach.} 
We can recover the relation for the evolution over a material volume such as a sum of water columns:
\begin{subequations}
  \begin{alignat}{2}
  \displaystyle 
  & \frac{d}{dt} \iiint_{\mathcal{V}_{s}} \rho A d\tau &&=
  \iiint_{\mathcal{V}_{s}} \rho \frac{d A}{dt} d\tau
  + \iiint_{\mathcal{V}_{s}}\mathbf{\nabla}.\left[ \rho A(  \mathbf{v}_{\Sigma}-\mathbf{v})\right] d\tau\\[4mm]
  & &&=
  \iiint_{\mathcal{V}_{s}} \rho \frac{d A}{dt} d\tau
  + \iint_{\mathcal{S}_{surf}} \rho A(\mathbf{v}_{\Sigma}-\mathbf{v}).\mathbf{n}_{\Sigma} dS_{\Sigma}
    \end{alignat}
\end{subequations}
\cite{griffies_fundamentals_2004} (page 140) further shows that, at the surface:
\begin{equation}
 \displaystyle
 \mathbf{n}_{\Sigma}.(\mathbf{v}_{\Sigma}-\mathbf{v})=\frac{h}{\sqrt{1+S^2}}\underbrace{\frac{d s}{dt}}_{=1/h\ w_s}
\end{equation}
and using $w_s=h\dot{s}$ thus:
\begin{subequations}
  \begin{alignat}{2}
  \displaystyle 
  & \frac{d}{dt} \iiint_{\mathcal{V}_{s}} \rho A d\tau &&=
  \iiint_{\mathcal{V}_{s}} \rho \frac{d A}{dt} d\tau
  +\iint_{\mathcal{S}_{surf}} \rho A  \underbrace{w_s}_{=0} dS
    \end{alignat}
\end{subequations}
